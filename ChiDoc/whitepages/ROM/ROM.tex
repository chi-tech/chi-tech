\documentclass[11pt,letterpaper,titlepage]{article}
\usepackage{fancyhdr}
\usepackage[left=0.75in, right=0.75in, bottom=1.0in]{geometry}
\usepackage{lastpage}
\usepackage{titleref}
\usepackage{booktabs}
\usepackage{appendix}
\appendixtitleon
\appendixtitletocon

\makeatletter

%================== List of figures and tables mods
\usepackage{tocloft}
\usepackage[labelfont=bf]{caption}

\renewcommand{\cftfigpresnum}{Figure\ }
\renewcommand{\cfttabpresnum}{Table\ }

\newlength{\mylenf}
\settowidth{\mylenf}{\cftfigpresnum}
\setlength{\cftfignumwidth}{\dimexpr\mylenf+1.5em}
\setlength{\cfttabnumwidth}{\dimexpr\mylenf+1.5em}



\newcommand{\half}{\frac{1}{2}}


%=================== Graphics
\usepackage{graphicx}
\usepackage[breakwords]{truncate}
\usepackage{float}
\usepackage{array}
\usepackage{amsmath}
\usepackage{mdframed}
\usepackage{fancyvrb}
\usepackage{float}
\usepackage{cancel}
\usepackage{amssymb}
\graphicspath{ {images/} }
\usepackage[usenames,dvipsnames,svgnames,table]{xcolor}
\usepackage[defaultlines=2,all]{nowidow}
\usepackage{listings}
\usepackage{color}
\definecolor{Brown}{cmyk}{0,0.81,1,0.60}
\definecolor{OliveGreen}{cmyk}{0.64,0,0.95,0.40}
\definecolor{CadetBlue}{cmyk}{0.62,0.57,0.23,0}
\usepackage{pdflscape}
\usepackage{relsize}
\usepackage{verbatim}
\usepackage{tabto}


%=================== Settings
\renewcommand{\baselinestretch}{1.2}
\definecolor{gray}{rgb}{0.4 0.4 0.4}
\newcommand{\stimes}{{\times}}
\setcounter{MaxMatrixCols}{20}

\begin{document}
\newcommand{\NSCDOCNUMBR}{NSC-REP-15-X}         %Put document number here
\newcommand{\NSCDOCSUBJT}{TECHNICAL REPORT: }   %Put document subject here
\newcommand{\NSCDOCTITLE}{Introduction to Reduced Order Modelling in $ChiTech$}       %Put document title here
\newcommand{\NSCDOCDATE} {August, 2018}    %Put document date here
\newcommand{\NSCDOCREV}  {Rev 1.0} %Put revision number here

\lstset{language=C++,frame=ltrb,framesep=4pt,basicstyle=\linespread{0.8} \small,
	keywordstyle=\ttfamily\color{OliveGreen},
	identifierstyle=\ttfamily\color{CadetBlue}\bfseries,
	commentstyle=\color{Brown},
	stringstyle=\ttfamily,
	showstringspaces=ture }


%################################# TITLE PAGE ########################
\begin{titlepage}
	\pagestyle{fancy}
	\vspace*{1.0cm}
	\centering
	%\includegraphics{NSC_Logo} \par
	\vspace{1cm}
	%\centering
	%{\Large\bfseries  \NSCDOCNUMBR   \par}
	\vspace{.25cm}
	%\centering
	{\Large\bfseries  \NSCDOCSUBJT \par} 
	{\Large\bfseries \NSCDOCTITLE  \par}
	\vspace{1cm}
	{\Large \NSCDOCDATE \par}
	\vspace{1.0cm}
	{\Large Jan Vermaak \par}
	{\Large \NSCDOCREV \par}
		
	\begin{comment}
	\renewcommand{\arraystretch}{2.0}
	\begin{tabular}{| m{2.5cm} | m{4.5cm} | m{4.5cm} |}
		\cline{2-3}
		\multicolumn{1}{c|}{} & \bfseries{Name} & \bfseries{Signature \& Date} \\ \hline
		\bfseries{Prepared} &     &     \\ \hline
		\bfseries{Reviewed} &     &     \\ \hline
		\bfseries{Reviewed} &     &     \\ \hline
	    \bfseries{Approved} &     &     \\ \hline
	\end{tabular} \par
	\end{comment}
	\begin{center}
		\begin{minipage}[c]{0.45\textwidth}
			\begin{figure}[H]
			
				\includegraphics[width=3in]{Logo2_Medium.png}
			\end{figure}
		\end{minipage}
	\end{center}
	\vspace{2cm}
	%NSC-FRM-15-1 Rev.1
\end{titlepage}


\pagestyle{fancy}
\rfoot{Page \thepage \ of \pageref{LastPage}}
%\cfoot{NSC-FRM-15-1 Rev.1}
\cfoot{}
\lfoot{\truncate{14cm}{\NSCDOCTITLE}}
\rhead{}
\chead{\currentname}
\lhead{}
\renewcommand{\footrulewidth}{0.4pt}
\tableofcontents
\addtocontents{toc}{~\hfill\textbf{Page}\par}

\listoffigures
%\listoftables
\chead{Contents}


\newpage
\chead{1 Introduction}
\section{Introduction}
Suppose we have heat diffusion in a 1D space:

\begin{equation}
-\nabla (k\nabla T) + s = 0
\end{equation}

Using a finite difference approximation of the derivatives we obtain:

\begin{equation*}
\begin{aligned}
-\frac{1}{\Delta x_i} 
\biggr [
k_{i+\half} \frac{T_{i+1}-T_{i}}{\Delta x_{i+\half}} -
k_{i-\half} \frac{T_{i}-T_{i-1}}{\Delta x_{i-\half}} 
\biggr] + s_i = 0
\end{aligned}
\end{equation*}

If the unknowns $T_i$ do not coincide with region boundary points, we obtain $k_{i\pm \half}$ from the harmonic mean:

$$
k_{i+\half} = \biggr( \frac{1}{k_i} + \frac{1}{k_{i+1}}     \biggr)^{-1} = 
\frac{k_i.k_{i+1}}{k_i+k_{i+1}}
$$

Alternatively, we can define the unknowns $T_i$ to coincide with region boundary points so that:

$$
k_{i+\half} = \frac{k_{i\to i+\half}+k_{i+\half\to i+1}}{2} = k_i
$$
Similarly:

$$
k_{i-\half} = k_{i-1}
$$

\noindent With equation 1 we can now construct a linear system $AT=b$,

\begin{equation*}
A=
\begin{bmatrix}
\frac{k_{i+1}}{\Delta x_i \Delta x_{i+\half}}
&-\frac{k_{i+1}}{\Delta x_i \Delta x_{i+\half}} &0 &\hdots &\hdots\\
\vdots &\ddots &\hdots &\hdots &\hdots\\
\vdots
&-\frac{k_{i-1}}{\Delta x_i \Delta x_{i-\half}} 
&\frac{k_{i-1}}{\Delta x_i \Delta x_{i-\half}} + \frac{k_{i+1}}{\Delta x_i \Delta x_{i+\half}}
&-\frac{k_{i+1}}{\Delta x_i \Delta x_{i+\half}}
&\hdots   \\
\end{bmatrix}
\end{equation*}
\newline
\noindent Furthermore we can write $A=f(\mu)$. Now suppose that we made enough calculations to support an emulated function which uses a set of basis vectors $(\Psi_i)_{(1\le i\le r)}$ and expansion coefficients $c(\mu)$ such that:

\begin{equation}
\begin{aligned}
T^r(\mu) &= 
[
\Psi_1^T \ \Psi_2^T \ .. \ \Psi_r^T 
]
\ c(\mu)\\
T^r(\mu) &= U^r c(\mu)
\end{aligned}
\end{equation}

Replacing $T$ in the linear system $AT=b$ and multiplying by $U^{r,T}$ from the left gives:

\begin{equation*}
content...
\end{equation*}







\newpage
\chead{References}
\begin{thebibliography}{1}
	\bibitem{ChengEtAl} {\em Two-Phase Flow Patterns and Flow-Pattern Maps: Fundamentals and Applications}, ASME, Applied Mechanics Reviews, 2008.
	\bibitem{RELAP5Vol6} Shieh, A.S., Ransom, V.H., Krishnamurthy, R., {\em RELAP5/MOD3 Code Manual Volume 6: Validation of numerical techniques in RELAP5/MOD3}, NUREG/CR-5535/Rev 1.
\end{thebibliography}





\end{document}