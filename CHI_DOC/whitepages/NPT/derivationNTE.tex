\documentclass[11pt,letterpaper,titlepage]{article}

%================== Document nomenclature
\newcommand{\DOCSUBJT}{Personal Notes: }   %Put document subject here
\newcommand{\DOCTITLE}{                      %Put document title here
	Derivation of the neutron transport equation, and the Discrete Ordinates Method
}       
\newcommand{\DOCDATE} {August, 2018}         %Put document date here
\newcommand{\DOCREV}  {Rev 1.00}             %Put revision number here

%================== Misc Settings
\usepackage{fancyhdr}
\usepackage[left=0.75in, right=0.75in, bottom=1.0in]{geometry}
\usepackage{lastpage}
\usepackage{titleref}
\usepackage{booktabs}
\usepackage{appendix}

\appendixtitleon
\appendixtitletocon

\makeatletter

%================== List of figures and tables mods
\usepackage{tocloft}
\usepackage[labelfont=bf]{caption}

\renewcommand{\cftfigpresnum}{Figure\ }
\renewcommand{\cfttabpresnum}{Table\ }

\newlength{\mylenf}
\settowidth{\mylenf}{\cftfigpresnum}
\setlength{\cftfignumwidth}{\dimexpr\mylenf+3.5em}
\setlength{\cfttabnumwidth}{\dimexpr\mylenf+1.5em}



%=================== Graphics
\usepackage{graphicx}
\usepackage[breakwords]{truncate}
\usepackage{float}
\usepackage{array}
\usepackage{amsmath}
\usepackage{mdframed}
\usepackage{fancyvrb}
\usepackage{float}
\usepackage{cancel}
\usepackage{amssymb}
\graphicspath{ {images/} }
\usepackage[usenames,dvipsnames,svgnames,table]{xcolor}
%\usepackage[defaultlines=2,all]{nowidow}
\usepackage{listings}
\usepackage{color}
\definecolor{Brown}{cmyk}{0,0.81,1,0.60}
\definecolor{OliveGreen}{cmyk}{0.64,0,0.95,0.40}
\definecolor{CadetBlue}{cmyk}{0.62,0.57,0.23,0}
\usepackage{pdflscape}
\usepackage{relsize}
\usepackage{verbatim}
\usepackage{tabto}
%\usepackage{upgreek}
\usepackage{enumitem}
%\usepackage{MnSymbol}% http://ctan.org/pkg/mnsymbol
\usepackage[pdf]{graphviz}
\usepackage[linesnumbered,lined,boxruled,algosection,commentsnumbered]{algorithm2e}
\usepackage{enumitem}

\definecolor{gray}{rgb}{0.4,0.4,0.4}
\definecolor{darkblue}{rgb}{0.0,0.0,0.6}
\definecolor{cyan}{rgb}{0.0,0.6,0.6}

\definecolor{ao(english)}{rgb}{0.0, 0.5, 0.0}

\newcommand{\xmltag}[1]{\textcolor{blue}{ \texttt{#1}} }
\newcommand{\xmloption}[1]{\textcolor{ao(english)}{ \texttt{#1}} }


\counterwithin{figure}{section}
\renewcommand{\thefigure}{\arabic{section}.\arabic{figure}}
%=================== Big cdot
\newcommand*\bigcdot{\mathpalette\bigcdot@{.5}}
\newcommand*\bigcdot@[2]{\mathbin{\vcenter{\hbox{\scalebox{#2}{$\m@th#1\bullet$}}}}}

\newcommand{\beq}{\begin{equation*}
\begin{aligned}}
\newcommand{\eeq}{\end{aligned}
\end{equation*}}

\newcommand{\beqn}{\begin{equation}
	\begin{aligned}}
\newcommand{\eeqn}{\end{aligned}
	\end{equation}}

%=================== Settings
\renewcommand{\baselinestretch}{1.2}
\definecolor{gray}{rgb}{0.4 0.4 0.4}
\newcommand{\stimes}{{\times}}

%================== Code syntax highlighting
\lstset{language=C++,frame=ltrb,framesep=2pt,basicstyle=\linespread{0.8} \small,
	keywordstyle=\ttfamily\color{OliveGreen},
	identifierstyle=\ttfamily\color{CadetBlue}\bfseries,
	commentstyle=\color{Brown},
	stringstyle=\ttfamily,
	showstringspaces=true,
	tabsize=2,}
	
%================== Section numbers with equation numbers
\numberwithin{equation}{section}


%================== Short \to arrow
\setlength{\medmuskip}{0mu}
%\newcommand{\tos}[1][3pt]{\mathrel{%
%   \hbox{\rule[\dimexpr\fontdimen22\textfont2-.2pt\relax]{#1}{.4pt}}%
%   \mkern-4mu\hbox{\usefont{U}{lasy}{m}{n}\symbol{41}}}}



\setlength\parindent{0pt}


\begin{document}

\begin{titlepage}
	\pagestyle{fancy}
	\vspace*{1.0cm}
	\centering
	\vspace{1cm}
	\vspace{.25cm}
	{\Large\bfseries  \DOCSUBJT \par} 
	{\Large\bfseries \DOCTITLE  \par}
	\vspace{1cm}
	{\Large \DOCDATE \par}
	\vspace{1.0cm}
	{\Large Jan Vermaak \par}
	{\Large \DOCREV \par}

\end{titlepage}	


\pagestyle{fancy}
\rfoot{Page \thepage \ of \pageref{LastPage}}
\cfoot{}
\lfoot{\truncate{14cm}{\DOCTITLE}}
\rhead{}
\chead{\currentname}
\lhead{}
\renewcommand{\footrulewidth}{0.4pt}

\newpage
\chead{Table of contents}
%\begin{comment}
\tableofcontents
\addtocontents{toc}{~\hfill\textbf{Page}\par}

\listoffigures
\listoftables

%\end{comment}
\chead{Contents}	

%#########################################################################
\newpage
\chead{Basics - The Boltzman Neutron Transport Equation}
\section{Basics - The Boltzman Neutron Transport Equation}
Let us denote the position of a neutron in space by $\mathbf{r}=[x \ y \ z]$ and the direction along which it is traveling by the normal vector $\hat{\Omega}$ such that

$$\hat{\Omega} = [\sin \theta . \cos \varphi \quad \sin \theta . \sin \varphi \quad \cos \theta]$$

\begin{figure}[h]
    \centering
    \includegraphics[width=0.5\linewidth]{PositionOmega}
    \caption{Orientation of direction in cartesian space.}
    \label{fig:positionomega}
\end{figure}

\noindent
Now we want to observe a control volume and the balance of neutrons within it. But first we have to define a few terms. First is the neutron density $n$, with units $[\frac{neutrons}{cm^3}]$, dependent on position, direction and energy, written as
$$n(r,\hat{\Omega},E,t).$$
\noindent For this scope we will drop the notion of time dependence. Neutron density can be multiplied by the velocity associated with its energy to determine the angular flux $\Psi$, written as
$$
\Psi (r,E,\hat{\Omega}) = v(E) n(r,E,\hat{\Omega}).
$$
\noindent This has units of $[\frac{neutrons}{cm^2 s}]$ and is essential to couple reactions to known cross-sections in the form of
$$\text{Reaction rate}=\Sigma_t \Psi$$

\noindent Let us now turn our attention to the balance of neutrons in a control volume.
\newpage


\begin{figure}[h]
    \centering
    \includegraphics[width=0.4\linewidth]{ControlVolume}
    \caption{Control volume for neutron balance.}
    \label{fig:controlvolume}
\end{figure}
\noindent
Consider the control volume as shown in figure \ref{fig:controlvolume}. The time rate of change of neutrons over the volume $\frac{dn}{dt}$ is given by

\begin{equation}
\begin{aligned}
\int_{dV}\frac{dn}{dt}.dr = 0
&= -\int_{S} \biggr[ \hat{n} \bigcdot \hat{\Omega} \Psi (r,E,\hat{\Omega})   \biggr]. dA
 -\int_{dV} \Sigma_a (r,E)\Psi(r,E,\hat{\Omega}).dr\\
&-\int_{dV}  \int_E \int_{4\pi} 
\Sigma_s (r,E{\to}E',\hat{\Omega} {\to} \hat{\Omega}') \Psi (r,E,\hat{\Omega}) 
.d\hat{\Omega}'.dE'.dr\\
&+\int_{dV}  \int_E \int_{4\pi}
\Sigma_s (r,E'{\to} E,\hat{\Omega}' {\to} \hat{\Omega}) \Psi (r,E',\hat{\Omega}') 
.d\hat{\Omega}'.dE'.dr\\
&+\frac{\chi (E)}{4\pi} 
\int_{dV}  \int_E \int_{4\pi} 
\Sigma_f (r,E',\hat{\Omega}') \Psi (r,E',\hat{\Omega}') 
.d\hat{\Omega}'.dE'.dr\\
&+\int_{dV} S(r,E,\hat{\Omega}).dr
\end{aligned}
\end{equation}
\newline
Gauss's divergence theorem on the streaming term:

\begin{equation*}
\int_{S} \biggr[ \hat{n} \bigcdot  \hat{\Omega}\Psi (r,E,\hat{\Omega})   \biggr]. dA = 
\int_{dV} \biggr[ \hat{\Omega}\nabla \Psi (r,E,\hat{\Omega})   \biggr]. dr
\end{equation*}
\newline
Now:
\begin{equation}
\begin{aligned}
\int_{dV}\frac{dN}{dt}.dr = 0
&= -\int_{dV} \biggr[ \hat{\Omega}\nabla \Psi (r,E,\hat{\Omega})   \biggr]. dr
-\int_{dV} \Sigma_a (r,E)\Psi(r,E,\hat{\Omega}).dr\\
&-\int_{dV}  \int_E \int_{4\pi} 
\Sigma_s (r,E{\to} E',\hat{\Omega} {\to} \hat{\Omega}') \Psi (r,E,\hat{\Omega}) 
.d\hat{\Omega}'.dE'.dr\\
&+\int_{dV}  \int_E \int_{4\pi}
\Sigma_s (r,E'{\to} E,\hat{\Omega}' {\to} \hat{\Omega}) \Psi (r,E',\hat{\Omega}') 
.d\hat{\Omega}'.dE'.dr\\
&+\frac{\chi (E)}{4\pi} 
\int_{dV}  \int_E \int_{4\pi} 
\Sigma_f (r,E',\hat{\Omega}') \Psi (r,E',\hat{\Omega}') 
.d\hat{\Omega}'.dE'.dr\\
&+\int_{dV} S(r,E,\hat{\Omega}).dr
\end{aligned}
\end{equation}

\noindent Dropping all the $\int_{dV}.dr$ gives:
\begin{equation}
\begin{aligned}
 0
&= -\hat{\Omega} \nabla \Psi (r,E,\hat{\Omega})
- \Sigma_a (r,E)\Psi(r,E,\hat{\Omega})\\
&-  \int_E \int_{4\pi} 
\Sigma_s (r,E{\to} E',\hat{\Omega} {\to} \hat{\Omega}') \Psi (r,E,\hat{\Omega}) 
.d\hat{\Omega}'.dE'\\
&+  \int_E \int_{4\pi}
\Sigma_s (r,E'{\to} E,\hat{\Omega}' {\to} \hat{\Omega}) \Psi (r,E',\hat{\Omega}') 
.d\hat{\Omega}'.dE'\\
&+\frac{\chi (E)}{4\pi} 
  \int_E \int_{4\pi} 
\Sigma_f (r,E',\hat{\Omega}') \Psi (r,E',\hat{\Omega}') 
.d\hat{\Omega}'.dE'\\
&+ S(r,E,\hat{\Omega})
\end{aligned}
\end{equation}
\newline
If we leave the NTE in this form we will do the scattering integral over all energy groups where neutrons are scattering to this group, and another integral over all energy groups where neutrons are scattering from this group to another. To reduce the complexity/difficulty of this we combine the scattering from group $E$ to $E'$ into the total removal cross-section:

\begin{equation*}
\begin{aligned}
\Sigma_t (r,E,\hat{\Omega})\Psi(r,E,\hat{\Omega}) &=  \Sigma_a (r,E)\Psi(r,E,\hat{\Omega})\\
&+  \int_E \int_{4\pi} 
\Sigma_s (r,E{\to} E',\hat{\Omega} {\to} \hat{\Omega}') \Psi (r,E,\hat{\Omega}) 
.d\hat{\Omega}'.dE'\\
\end{aligned}
\end{equation*}
\newline
Which gives us the base NTE:

\begin{equation}
\begin{aligned}
\hat{\Omega}\nabla \Psi (r,E,\hat{\Omega}) 
+\Sigma_t (r,E)\Psi(r,E,\hat{\Omega})
&=  \int_E \int_{4\pi}
\Sigma_s (r,E'{\to} E,\hat{\Omega}' {\to} \hat{\Omega}) \Psi (r,E',\hat{\Omega}') 
.d\hat{\Omega}'.dE'\\
&+\frac{\chi (E)}{4\pi} 
  \int_E \int_{4\pi} 
\Sigma_f (r,E',\hat{\Omega}') \Psi (r,E',\hat{\Omega}') 
.d\hat{\Omega}'.dE'\\
&+ S(r,E,\hat{\Omega})
\end{aligned}
\end{equation}
\newline
\newline
If we want to remove more complexity we can combine the fission source term into the source to get the Neutron Transport Equation in its most basic form:

\begin{equation} \label{eq:baseNTE}
\begin{aligned}
\biggr(\hat{\Omega}\nabla +\Sigma_t (r,E)\biggr) \Psi(r,E,\hat{\Omega})
&=  \int_E \int_{4\pi}
\Sigma_s (r,E'{\to} E,\hat{\Omega}' {\to} \hat{\Omega}) \Psi (r,E',\hat{\Omega}') 
.d\hat{\Omega}'.dE'\\
&+ S(r,E,\hat{\Omega})
\end{aligned}
\end{equation}

\newpage
\chead{Spherical harmonics and Legendre Polynomials}
\section{Spherical harmonics and Associated Legendre Polynomials}
The entry point for a person researching spherical harmonics is inevitably the wikipedia and wolfram-alpha websites describing the complex valued spherical harmonics. However, from various sources it is evident that we can compute the same expansion by using the real forms of the sperical harmonics, called the \textbf{tesseral spherical harmonics}. This expansion is done in the form

\beqn 
f(\varphi,\theta) = \sum_{\ell = 0}^\infty \sum_{m=-\ell}^{\ell} \frac{2\ell+1}{4\pi} f_{\ell m} Y_{\ell m}(\varphi,\theta)
\eeqn 

where
$$
f_{\ell m} = \int_{0}^{2\pi} \int_0^\pi f(\varphi,\theta)Y_{\ell m}(\varphi,\theta).sin\theta.d\theta.d\varphi
$$
and

\beqn
Y_{\ell m} (\theta, \varphi )=
\begin{cases}
 \sqrt(2)\sqrt{  \frac{(\ell-|m|)!}{(\ell+|m|)!}}P_{\ell}^{|m|}(\cos\theta)sin\ {|m|\varphi}
& \text{if } m < 0 \\
 P_{\ell}^{m}(cos\theta) & \text{if } m = 0 \\
 \sqrt(2)\sqrt{    \frac{(\ell-m)!}{(\ell+m)!}}P_{\ell}^{m}(\cos\theta)cos\ {m\varphi}
& \text{if } m \ge 0 \\
\end{cases}
\eeqn
\newline
And finally the polynomials called the \textbf{Associated Legendre Polynomials} are determined from

\beqn \label{eq:AssLegendre}
P_0^0 &= 1, \quad \quad \quad
P_1^{0} = x, \\
P_\ell^\ell &= - (2\ell-1) \sqrt{1-x^2} \ P_{\ell-1}^{\ell-1}(x) \quad \text{ and}\\
(\ell - m)
P_\ell^m &= (2\ell-1)x \ P_{\ell-1}^m(x) - (\ell+m -1)P_{\ell-2}^m (x).
\eeqn
\newline
It is important to note that from the basic definition of the associated Legendre Polynomials

\beq 
P_\ell^m(x) = (-1)^m (1-x^2)^{m/2} \frac{d^m}{dx^m} (P_\ell (x))
\eeq 
\newline
we can see that the associated Legendre Polynomials equates to the ordinary Legendre Polynomials when $m=0$.

\newpage
\chead{The scattering Kernel and Legendre expansion}
\section{The scattering Kernel and Legendre expansion}
The scattering cross-section of materials are normally available as a function of energy only. It is up to the user of the data to add the appropriate scattering kernels for which the scattering angles are a function of the masses of the neutron and its colliding nucleus as well as the energy. But first let us express the scattering terms in a Kernel fashion. We start with the scattering source term as is appears in equation \ref{eq:baseNTE}:

\begin{align*}
\int_E \int_{4\pi}
\Sigma_s (r,E'{\to} E,\hat{\Omega}' {\to} \hat{\Omega}) \Psi (r,E',\hat{\Omega}') 
.d\hat{\Omega}'.dE'.\\
\end{align*}

We then split the macroscopic scattering cross-section into an energy dependent but not angularly dependent term, $\Sigma_s (r,E')$, and a separate kernel $K(E'{\to}E,\hat{\Omega}'{\to}\hat{\Omega})$ as 

\begin{align*}
\int_E \int_{4\pi}
\Sigma_s (r,E').K(E'{\to}E,\hat{\Omega}'{\to}\hat{\Omega}).\Psi (r,E',\hat{\Omega}') 
.d\hat{\Omega}'.dE'.\\
\end{align*}
\newline
Naturally, we require 
\begin{align} \label{eq:kernelIntegralUnity}
\int_E\int_{4\pi}
K(E'{\to}E,\hat{\Omega}'{\to}\hat{\Omega}) 
.d\hat{\Omega}.dE=1.
\end{align}
\newline
Now we expand this Kernel using Legendre polynomials but before we do this we have to handle the notion of $\hat{\Omega}'{\to}\hat{\Omega}$. We redefine the kernel as being dependent on the scattering angle $\theta_s$ where

\begin{align*}
cos\theta_s = \hat{\Omega}'\bullet\hat{\Omega} = \mu
\end{align*}

\begin{align*}
K(cos\theta_s,E'{\to}E) &= \sum_{\ell=0}^\infty \frac{2\ell+1}{2} P_\ell (cos\theta_s) K_\ell (E'{\to}E) \text{ or } \\
K(\mu,E'{\to}E) &= \sum_{\ell=0}^\infty \frac{2\ell+1}{2} P_\ell (\mu) K_\ell (E'{\to}E)
\end{align*}

\noindent This expansion is over the entire radial cone $2\pi$ and therefore we include a normalization factor of $\frac{1}{2\pi}$. The scattering term therefore becomes

\begin{align} \label{eq:scatterKernelLegendre}
\Sigma_s (r,E')K(E'{\to}E,\hat{\Omega}'{\to}\hat{\Omega}) = \Sigma_s (r,E')
\sum_{\ell=0}^\infty \frac{2\ell+1}{4\pi} P_\ell (cos\theta_s) K_\ell (E'{\to}E).
\end{align} 



\newpage
\chead{Expansion of angular variables in Spherical Harmonics}
\section{Expansion of angular variables in Spherical Harmonics}
We can expand the angular flux that appears in the scattering source into spherical harmonics as
\begin{align} \label{eq:angularFlux}
\Psi(r,E',\hat{\Omega}') = \sum_{\ell=0}^{\infty}\sum_{m=-\ell}^{\ell} 
\frac{2\ell+1}{4\pi}
  \phi_{\ell m}(r,E') Y_{\ell m}(\hat{\Omega}' ),
\end{align}
where
\begin{align*} 
\phi_{\ell m}(r,E')
&=\int _{\hat{\Omega} }\Psi(r,E',\hat{\Omega} )\,Y_{\ell m}(\hat{\Omega})\,d\hat{\Omega}, 
\end{align*}
however, we never calculate $\phi_{\ell m}$ with this integral since $\Psi$ is still an unknown. Instead we calculate it from a set of linear equations as we will observe later. Now, if we plug equation \ref{eq:scatterKernelLegendre} and \ref{eq:angularFlux} into equation \ref{eq:baseNTE} we get

\begin{equation} \label{eq:baseNTEwithSPH_andLegendre}
\begin{aligned}
&\biggr(\hat{\Omega}\nabla +\Sigma_t (r,E)\biggr) \Psi (r,E,\hat{\Omega})\\
&=  \int_E \int_{4\pi} \biggr[
\Sigma_s (r,E')
\sum_{\ell=0}^\infty \frac{2\ell+1}{4\pi} P_{\ell} (cos\theta_s) K_{\ell} (E'{\to}E) \sum_{\ell'=0}^{\infty}\sum_{m'=-\ell'}^{\ell'} \frac{2\ell'+1}{4\pi}\phi_{\ell' m'} (r,E') Y_{\ell' m'} (\hat{\Omega}' ) \biggr]
.d\hat{\Omega}'.dE'\\
&+ S (r,E,\hat{\Omega}) .
\end{aligned}
\end{equation}
\newline
Now, the addition theorem of spherical harmonics states
\begin{align} \label{eq:additionTheoremSPH}
P_{\ell} (cos\theta_s)=
P_{\ell} (\hat{\Omega}'\cdot\hat{\Omega}) = \sum_{m=-\ell}^{\ell} Y_{\ell m}(\hat{\Omega}) Y_{\ell m}(\hat{\Omega}'),
\end{align}
\noindent which if added into equation \ref{eq:baseNTEwithSPH_andLegendre} gives

\begin{equation} \label{eq:baseNTEwithSPH_andLegendre_andaddTheorem}
\begin{aligned}
&\biggr(\hat{\Omega}\nabla +\Sigma_t (r,E)\biggr) \Psi (r,E,\hat{\Omega})\\
&=  \int_E \int_{4\pi} \biggr[
\Sigma_s (r,E')
\sum_{\ell=0}^\infty \frac{2\ell+1}{4\pi} \sum_{m=-\ell}^{\ell} Y_{\ell m}(\hat{\Omega}) Y_{\ell m}(\hat{\Omega}') K_{\ell} (E'{\to}E) \sum_{\ell'=0}^{\infty}\sum_{m'=-\ell'}^{\ell'} \frac{2\ell'+1}{4\pi}\phi_{\ell' m'} (r,E') Y_{\ell' m'} (\hat{\Omega}' ) \biggr]
.d\hat{\Omega}'.dE'\\
&+ S (r,E,\hat{\Omega}).
\end{aligned}
\end{equation}
\newline
Now using the orthogonality of spherical harmonics

\begin{align} \label{eq:orthogonalitySPH}
\int_{4\pi} Y_{l'm'}(\hat{\Omega}')Y_{\ell m} (\hat{\Omega}').d\hat{\Omega} =
\frac{4\pi}{2\ell' +1} \delta_{ll'} \delta_{mm'},
\end{align}

\noindent equation \ref{eq:baseNTEwithSPH_andLegendre_andaddTheorem} becomes

\begin{equation} 
\begin{aligned}
&\biggr(\hat{\Omega}\nabla +\Sigma_t (r,E)\biggr)  \Psi (r,E,\hat{\Omega})\\
&= \sum_{\ell=0}^{\infty}\sum_{m=-\ell}^{\ell} \frac{2\ell+1}{4\pi}Y_{\ell m}(\hat{\Omega})
\biggr[ \int_E 
\Sigma_s (r,E')
  K_\ell (E'{\to}E)  \phi_{\ell m} (r,E').dE' 
 \biggr] \\
&+  S (r,E,\hat{\Omega}).
\end{aligned}
\end{equation}
\newline
In practise the seperation of the scattering cross-section from the scattering kernel is unnecessary and thus we can simply the above equation by definining 

\beq 
\Sigma_{s\ell} (E') = \Sigma_s (r,E')
  K_\ell (E'{\to}E)
\eeq 
\newline
This leaves us with the convenient form
\begin{equation} 
\begin{aligned}
&\biggr(\hat{\Omega}\nabla +\Sigma_t (r,E)\biggr)  \Psi (r,E,\hat{\Omega})\\
&= \sum_{\ell=0}^{\infty}\sum_{m=-\ell}^{\ell} \frac{2\ell+1}{4\pi}Y_{\ell m}(\hat{\Omega})
\biggr[ \int_E 
\Sigma_{s\ell} (E')  \phi_{\ell m} (r,E').dE' 
 \biggr] \\
&+  S (r,E,\hat{\Omega}).
\end{aligned}
\end{equation}


\vspace{1cm}
\chead{Multi-group approximation of the energy}
\section{Multi-group approximation of the energy}
Pretty easy to see:


\begin{equation} \label{eq:multiGroupNTE}
\begin{aligned}
&\biggr(\hat{\Omega}\nabla +\Sigma_{tg} (r)\biggr)  \Psi_g (r,\hat{\Omega})\\
&= \sum_{\ell=0}^{\infty}\sum_{m=-\ell}^{\ell}\frac{2\ell+1}{4\pi} Y_{\ell m}(\hat{\Omega})
\biggr[ \sum_{g'=0}^{G}
\Sigma_{s\ell,g'{\to}g} (r)
  \phi_{g'\ell m} (r)
\biggr]\\
&+  S_g (r,\hat{\Omega}).
\end{aligned}
\end{equation}







\newpage
\chead{The Discrete Ordinates Method ($S_N$ Method)}
\section{The Discrete Ordinates Method ($S_N$ Method)}
For a particular direction $\hat{\Omega}_n$ we write

$$\Psi_g (r,\hat{\Omega}_n) = \Psi_{gn}(r)$$

The multi-group neutron transport equation (i.e. equation \ref{eq:multiGroupNTE}) now becomes the angular neutron transport equation


\begin{equation} \label{eq:angularNTE}
\begin{aligned}
&\biggr(\hat{\Omega}_n\nabla +\Sigma_{tg} (r)\biggr)  \Psi_{gn} (r).dr\\
&= \sum_{\ell=0}^{\infty}\sum_{m=-\ell}^{\ell} \frac{2\ell+1}{4\pi}Y_{\ell m}(\hat{\Omega}_n)
\biggr[ \sum_{g'=0}^{G}
\Sigma_{s\ell,g'{\to}g} (r)
\phi_{g'\ell m} (r)
\biggr] .dr\\
&+  S_{gn} (r).dr.
\end{aligned}
\end{equation}
\newline
Recall that an angular function expanded using real spherical harmonics is denoted as

$$f(\theta,\varphi) = \sum_{\ell=0}^\infty \sum_{m=-\ell}^\ell  \frac{2\ell+1}{4\pi} f_{\ell m}Y_{\ell m} (\theta,\varphi),$$

\noindent where

$$f_{\ell m} = \int_0^\pi \int_0^{2\pi} f(\theta,\varphi) Y_{\ell m}(\theta,\varphi)    .d\varphi.d\theta$$

\noindent Similarly,

\begin{equation*}
\phi_{g\ell m}(r) = \int_{4\pi} \Psi_g(r,\hat{\Omega}) Y_{\ell m}(\hat{\Omega}).d\hat{\Omega}.
\end{equation*}
\newline
\noindent Now the fundamental trick is to approximate this integral with a \textbf{quadrature rule} or a set of quadrature rules such that

\begin{equation*}
\int_{4\pi} \Psi_g(r,\hat{\Omega}) Y_{\ell m}(\hat{\Omega}).d\hat{\Omega} \approx \sum_{n=0}^{N} w_n . \Psi_g(r,\hat{\Omega}_n) Y_{\ell m}(\hat{\Omega}_n)
\end{equation*}
\begin{equation}
\therefore 
\int_{4\pi} \Psi_g(r,\hat{\Omega}) Y_{\ell m}(\hat{\Omega}).d\hat{\Omega} \approx \sum_{n=0}^{N} w_n . \Psi_{gn}(r) Y_{\ell m}(\hat{\Omega}_n),
\end{equation}
\newline
Where the weights $w_n$ and associated directions $\hat{\Omega}_n$ are particular to the quadrature rule (or rule set) used.
Plugging this into equation \ref{eq:angularNTE} we get the Discrete Ordinates ($S_n$) equations

\begin{equation} \label{eq:SNequations}
\begin{aligned}
&\biggr(\hat{\Omega}_n\nabla +\Sigma_{tg} (r)\biggr)  \Psi_{gn} (r)\\
&= \sum_{\ell=0}^{\infty}\sum_{m=-\ell}^{\ell} \frac{2\ell+1}{4\pi}Y_{\ell m}(\hat{\Omega}_n)
\biggr[ \sum_{g'=0}^{G}
\Sigma_{s\ell,g'{\to}g} (r)
\biggr(
\sum_{n'=0}^{N} w_{n'} . Y_{\ell m}(\hat{\Omega}_{n'}).\Psi_{g'n'}(r) 
\biggr)
\biggr]\\
&+  S_{gn} (r).
\end{aligned}
\end{equation}



\vspace{1cm}
\subsection{Recasting the moment indices}
At this moment it is necessary to think about how to solve this problem. The indices of the spherical harmonics ($Y_{\ell m}$) is bothersome since it is not linear and therefore we can recast them. Let us suppose we have the regular spherical harmonics $Y_{lm_{true}}$, we propose $Y_{m}$ with only one index such that:

\begin{equation*}
\begin{matrix}
\text{when }\ell=0,  &Y_0 = Y_{0,0} \\
\text{when }\ell=1,  &Y_1 = Y_{1,-1} &Y_2 = Y_{1,0} &Y_3 = Y_{1,1} \\
\text{when }\ell=2,  &Y_4 = Y_{2,-2} &Y_5 = Y_{2,-1} &Y_6 = Y_{2,0} &Y_7 = Y_{2,1} & Y_8 = Y_{2,2}\\
        &               &\vdots \\
        &               &\text{ and so forth}.
\end{matrix}
\end{equation*}

\noindent The maximum index per $\ell$ follows the sequence

\begin{align}
\begin{matrix}
\ell=0       &\ell=1      &\ell=2                      &\hdots     &\ell=L \\
0            &0+(2+1) & 0+(2 +1)+(4+1)     &\hdots     &0+(2 +1)+(4+1) +...+(2L+1)
\end{matrix}
\end{align}
\newline
\noindent which can be expressed as a series that has the form 
\begin{align}
m_{max}=\sum_{\ell=1}^L (2\ell+1),
\end{align}
but when expanded in a forward and backward sense such that 
\begin{align*}
\sum_{\ell=1}^L (2\ell+1)  &= (2+1)+(4+1)+(6+1)+...+(2(L-1)+1)+(2L+1) \\
\text{and   }
\sum_{\ell=1}^L (2\ell+1)  &= (2L+1)+(2(L-1)+1))+...+(6+1)+(4+1)+(2+1), 
\end{align*}
\noindent we can observe a pattern when adding these sequences together:
\begin{align*}
2\sum_{\ell=1}^L (2\ell+1) &= (2L+4)+(2L+4)+...+(2L+4)+(2L+4)+(2L+4) \\
&= L(2L+4) 
\end{align*}
\noindent
Therefore this series reduces to
\begin{align}
m_{max} = L(L+2)
\end{align}
\newline
\noindent Therefore the maximum $m$ for a given $\ell$ is $\ell(\ell+2)$ and corresponds to $m_{true}=+\ell$. This leads us to the index of any $m=\ell(\ell+2)+m_{true}-\ell$ with the test shown in figure \ref{fig:indices} below.


\begin{figure}[h]
    \centering
    \includegraphics[width=0.5\linewidth]{jindex.png}
    \caption{Indices of $Y_j$ as a subset of $Y_{\ell m}$.}
    \label{fig:indices}
\end{figure}

\noindent With this convention we can recast equation \ref{eq:SNequations} using 

\begin{align*}
\phi_{mg'} = \sum_{n'=0}^{N}
w_{n'}
Y_{m}(\hat{\Omega}_n')
. \Psi_{g'n'}(r),
\end{align*}
and the notion that $\phi_{mg'} = \phi_{\ell m_{true}g'}$ for $m=\ell(\ell+2)+m_{true}-\ell$, as well as $\Sigma_{sm,g'{\to}g} = \Sigma_{sm_{true},g'{\to}g}$, to

\begin{equation} \label{eq:SNequationsSingleMOm}
\begin{aligned}
&\biggr(\hat{\Omega}_n\nabla +\Sigma_{tg} (r)\biggr)  \Psi_{gn} (r)\\
&=       
\sum_{m=0}^{m_{max}}
M_{m}(\hat{\Omega}_n)
\sum_{g'=0}^{G} \biggr[  
\Sigma_{sm,g'{\to}g} (r)
. \biggr(
\sum_{n'=0}^{N} w_{n'} . Y_{m}(\hat{\Omega}_{n'}).\Psi_{g'n'}(r) 
\biggr)
\biggr]\\
&+  S_{gn} (r).
\end{aligned}
\end{equation}
\newline
Where $M_m$ is developed from each $\ell,m_{true}$ pair to get
\begin{equation*}
\begin{aligned}
m=\ell(\ell+1)+ m_{true}\\
M_m(\hat{\Omega}_n) = \frac{2\ell+1}{4\pi}Y_{\ell m}
\end{aligned}
\end{equation*}

We now define the notation $M_m(\hat{\Omega}_n)=M_{mn}$ and $Y_m (\hat{\Omega}_n) = Y_{mn}$, drop the notation $(r)$ for simplicity and rearrange the terms in equation \ref{eq:SNequationsSingleMOm} to get

\begin{equation} \label{eq:SNequationsSingleMOmReArr}
\begin{aligned}
&\biggr(\hat{\Omega}_n\nabla +\Sigma_{tg} \biggr)  \Psi_{gn} \\
&=    \sum_{g'=0}^{G} \biggr[    
\sum_{m=0}^{m_{max}}
M_{mn}
. 
\Sigma_{sm,g'{\to}g} 
. \biggr(
\sum_{n'=0}^{N} w_{n'} . Y_{mn'}.\Psi_{g'n'}
\biggr)
\biggr]\\
&+  S_{gn} (r).
\end{aligned}
\end{equation}


\vspace{1.5cm}
\subsection{Operator form}
The streaming and removal terms can be denoted as $L$ such that

\begin{align*}
\biggr(\hat{\Omega}_n\nabla +\Sigma_{tg} \biggr)  \Psi_{gn}  = \mathbf{L}_{gn}\Psi_{gn} 
\end{align*}

\noindent For the scattering source term we see that we have a Discrete-to-moment operator $\mathbf{D}_{mn}$, resulting from quadrature rule of integration over angle, such that

\begin{align*}
\mathbf{D}_m = \sum_{n'=0}^{N} w_{n'} . Y_{m n'}.\Psi_{g'n'}
= 
\begin{bmatrix}
w_0 Y_{m0} &\hdots  &w_{N} Y_{mN}\\
\end{bmatrix}
\begin{bmatrix}
\Psi_{g'0} \\
\vdots    \\
\Psi_{g'N}
\end{bmatrix}
=\mathbf{D}_m\Psi_{g'}(r) =\phi_{mg'}.
\end{align*}
\begin{align*}
\therefore 
\mathbf{D}_{mn} \begin{bmatrix}
\Psi_{g'0} \\
\vdots    \\
\Psi_{g'N}
\end{bmatrix} = \mathbf{\phi}_{g'}
\end{align*}
\newline
We also have the scattering operator $\mathbf{S}_{m,g'{\to}g}$ such that
\begin{align*}
\mathbf{S}_{m,g'{\to}g} \mathbf{D}_m\Psi_{g'} &= 
\Sigma_{sm,g'{\to}g} (r)
. \biggr(
\sum_{n'=0}^{N} w_{n'} . Y_{m n'}.\Psi_{g'n'}(r) 
\biggr)\\
&=
\begin{bmatrix}
\Sigma_{s,m,g'{\to}g} \\
\end{bmatrix}
\begin{bmatrix}
w_0 Y_{m 0} &\hdots  &w_{N} Y_{m N}\\
\end{bmatrix}
\begin{bmatrix}
\Psi_{g'0} \\
\vdots    \\
\Psi_{g'N}
\end{bmatrix} \\
&=\mathbf{S}_{m,g'{\to}g} \phi_{mg'}
\end{align*}
and the Moment-to-Discreet operator $\mathbf{M}_{mn}$ such that 
\begin{align*}
\biggr(\mathbf{M}_{mn}\mathbf{S}_{m,g'{\to}g} \mathbf{D}_m\Psi_{g'}\biggr)= 
\begin{bmatrix}
M_{mn}
\end{bmatrix}
\begin{bmatrix}
\Sigma_{s,m,g'{\to}g} \\
\end{bmatrix}
\begin{bmatrix}
w_0 Y_{m0} &\hdots  &w_{N} Y_{mN}\\
\end{bmatrix}
\begin{bmatrix}
\Psi_{g'0} \\
\vdots    \\
\Psi_{g'N}
\end{bmatrix}.
\end{align*}
\newline
\noindent The summation over moments now becomes:

\begin{align*}
&\sum_{m=0}^{m_{max}}
Y_{mn}
. 
\Sigma_{sm,g'{\to}g} (r)
. \biggr(
\sum_{n'=0}^{N} w_{n'} . Y_{m^*n'}.\Psi_{g'n'}(r) 
\biggr)\\
&=\sum_{m=0}^{m_{max}}\biggr(\mathbf{M}_{mn}\mathbf{S}_{m,g'{\to}g} \mathbf{D}_m\Psi_{g'}\biggr)
\\
&=
\begin{bmatrix}
\mathbf{M}_{0n}\mathbf{S}_{0,g'{\to}g} \mathbf{D}_0
&\hdots
&\mathbf{M}_{m_{max}n}\mathbf{S}_{m_{max},g'{\to}g} \mathbf{D}_{m_{max}}
\end{bmatrix}
\begin{bmatrix}
\Psi_{g'0} \\
\vdots    \\
\Psi_{g'N}
\end{bmatrix}
\\
&=
\begin{bmatrix}
\mathbf{M}_{0n}\mathbf{S}_{0,g'{\to}g} 
&\hdots
&\mathbf{M}_{m_{max}n}\mathbf{S}_{m_{max},g'{\to}g} 
\end{bmatrix}
\begin{bmatrix}
\mathbf{D}_0 \\
\vdots    \\
\mathbf{D}_{m_{max}}
\end{bmatrix}
\begin{bmatrix}
\Psi_{g'0} \\
\vdots    \\
\Psi_{g'N}
\end{bmatrix}
\\
&=
\begin{bmatrix}
\mathbf{M}_{0n}
&\hdots
&\mathbf{M}_{m_{max}n}
\end{bmatrix}
\begin{bmatrix}
\mathbf{S}_{0,g'{\to}g} \\
\vdots    \\
\mathbf{S}_{m_{max},g'{\to}g} 
\end{bmatrix}
\begin{bmatrix}
\mathbf{D}_0 \\
\vdots    \\
\mathbf{D}_{m_{max}}
\end{bmatrix}
\begin{bmatrix}
\Psi_{g'0} \\
\vdots    \\
\Psi_{g'N}
\end{bmatrix}
\\
&=\mathbf{M}_n \mathbf{S}_{g'{\to}g} \mathbf{D} \Psi_{g'}
\end{align*}
\newline
Reducing our transport equation to

\begin{align}
\mathbf{L}_{gn}\Psi_{gn} 
= \sum_{g'=0}^{G} \biggr[
\mathbf{M}_n \mathbf{S}_{g'{\to}g} \mathbf{D} \Psi_{g'}
\biggr]
+s_{gn}
\end{align}

And then to
\begin{align}
\mathbf{L}_{gn}\Psi_{gn} 
= \mathbf{M}_n \mathbf{S} \mathbf{D} \Psi
+s_{gn}
\end{align}

If we then sum over all angles we get:

\begin{align} \label{eq:NTEoperatorform}
\mathbf{L}\Psi 
= \mathbf{M} \mathbf{S} \mathbf{D} \Psi
+s
\end{align}


\subsection{No scattering - Computing the fixed source}
With no scattering the discrete problem to solve per cell is 
\begin{align*}
\mathbf{L}_{gn}\Psi_{gn} 
= 
s_{gn} \\
\Psi_{gn} 
= 
\mathbf{L}_{gn}^{-1} s_{gn} 
\end{align*}
and can be solved by sweeping along the direction of each angle. Suppose we store the angular flux then per cell and sweep through all angles. We can then obtain the scalar flux from
\begin{align*}
\phi = \mathbf{D}\Psi
= 
\mathbf{D}\mathbf{L}^{-1} s.
\end{align*}
However, the storage of the angular flux is prohibitively expensive, and therefore we have to look at the moment-by-moment accumulation of the scalar flux moments. First we can look again at 
\begin{align*}
\sum_{n=0}^{N} w_{n} . Y_{mn}.\Psi_{gn}
= 
\begin{bmatrix}
w_0 Y_{m0} &\hdots  &w_{N} Y_{mN}\\
\end{bmatrix}
\begin{bmatrix}
\Psi_{g0} \\
\vdots    \\
\Psi_{gN}
\end{bmatrix}
=\mathbf{D}_m\Psi_{g}(r) =\phi_{mg}.
\end{align*}
This equation hints at the possibility of \textit{accumulating} the scalar flux by starting with the first angle
$$\phi_{mg}^{new}=0+w_0 Y_{m0}\Psi_{g0}$$
and thereafter
$$\phi_{mg}^{new}=\phi_{mg}^{old}+w_n Y_{m n}\Psi_{gn}.$$
In this fashion we only ever have to store the scalar flux moments in each cell. For fixed source problems, it can be advantageous to call this step the \textit{fixed source caculation} since it does not change when adding scattering or reaction based neutron source.


\subsection{With scattering - Computing the scattering source (Richardson Iteration)}
The addition of a scattering source, or for that matter a reaction based neutron source like fission, transforms the problem into an implicit system of equations to solve. The reason for this is that the scattering source requires a flux driver ... one we don't have at the start of the problem. To this end we manipulate our transport equation as follows:
\begin{align*}
\mathbf{L}\Psi 
&= \mathbf{M} \mathbf{S} \mathbf{D} \Psi
+s \\
\Psi &= \mathbf{L}^{-1} \mathbf{MSD}\Psi + \mathbf{L}^{-1} s \\
 \mathbf{D}\Psi &=  \mathbf{D}\mathbf{L}^{-1} \mathbf{MSD}\Psi +  \mathbf{D}\mathbf{L}^{-1} s \\
 \therefore
\phi &= \mathbf{D}\mathbf{L}^{-1} \mathbf{MS}\phi +  \mathbf{D}\mathbf{L}^{-1} s
\end{align*}

One solution to this form of the problem is: when we are computing $\phi$ at iteration $\ell$ (i.e. $\phi^{(\ell)}$) we can approximate the scattering source from the previous iteration's $\phi$ (i.e. $\phi^{(\ell-1)}$) which leads to 
\begin{align*}
\phi^{(\ell)} &= \mathbf{D}\mathbf{L}^{-1} \mathbf{MS}\phi^{(\ell-1)} +  \mathbf{D}\mathbf{L}^{-1} s.
\end{align*}
This form is known as \textbf{Richardson Iteration}.


\newpage
\chead{Piece-wise Linear Shape functions in the Finite Element application}
\section{Piece-wise Linear Shape functions in the Finite Element application}
The lower triangular operator $\mathbf{L}$ in equation \ref{eq:NTEoperatorform} is the operator to a hyperbolic partial differential equation for which the method of choice to discretize space is the Discontinuous Galerkin method (DG). There are two well known methods for constructing the basis functions for the DG-method, one is the use of shape functions defined on a node and connecting to all nodes of subscribing cells and when defined as linear functions these are called Piece-Wise Linear basis functions. The other method is the use of Locally Discontinuous basis functions which will not be discussed here. We shall refer to the Discontinous Galerkin method with the use of Piece-wise Linear basis functions as PWLD for further discussions.
\newline 
\newline
To see where this fits into the finite element method we consider a simple first order partial differential equation
\beqn 
\Omega \bigcdot \nabla \Psi + \sigma \Psi= q.
\eeqn 

For each node we multiply by the (not yet defined) trial space $\tau_i$ and require that 

\beq
\int_V \biggr[ \tau_i \Omega \bigcdot \nabla \Psi + \sigma \tau_i \Psi  -\tau_i q \biggr].dV = 0.
\eeq
\beqn
\therefore
\int_V\tau_i \Omega \bigcdot  \nabla \Psi .dV+ \int_V \sigma \tau_i \Psi.dV  -\int_V\tau_i q .dV= 0.
\eeqn 
Applying integration by parts for the first term results in 

\beqn
\int_V \Omega \bigcdot \nabla (\tau_i \Psi).dV  - \int_V  \Psi \Omega \bigcdot \nabla \tau_i.dV + \int_V \sigma \tau_i \Psi.dV-\int_V\tau_i q .dV= 0
\eeqn 

where we can apply Gauss's divergence theorem to the first term to obtain

\beqn
\int_{\partial V} (\Omega \bigcdot \hat{n}) \tau_i \Psi.dS  - \int_V \Psi \Omega \bigcdot  \nabla \tau_i.dV+ \int_V \sigma \tau_i \Psi.dV-\int_V\tau_i q .dV= 0.
\eeqn 

We now apply an upwinding scheme to the boundary integral 

\beq
\int_{\partial V} (\Omega \bigcdot \hat{n}) \tau_i \tilde{\Psi}.dS  - \int_V \Psi \Omega \bigcdot  \nabla \tau_i.dV+ \int_V \sigma \tau_i \Psi.dV-\int_V\tau_i q .dV= 0.
\eeq 
where
\beqn
\tilde{\Psi} =
\begin{cases}
\Psi_{within cell} \quad \quad &\text{if } \hat{n}\bigcdot \Omega > 0 \\
\Psi_{upwind cell} \quad \quad &\text{if } \hat{n}\bigcdot \Omega < 0 \\
\end{cases}
\eeqn 

We now apply integration again to the second term

\beq
\int_{\partial V} (\Omega \bigcdot \hat{n}) \tau_i \tilde{\Psi}.dS  
- \int_V \Omega \bigcdot  \nabla (\tau_i \Psi ) .dV 
+\int_V \tau_i \Omega \bigcdot  \nabla \Psi .dV+ \int_V \sigma \tau_i \Psi.dV-\int_V\tau_i q .dV= 0.
\eeq 

and then Gauss's divergence theorem again on the second term of this

\beq
\int_{\partial V} (\Omega \bigcdot \hat{n}) \tau_i (\tilde{\Psi} - \Psi).dS  
+\int_V \tau_i \Omega \bigcdot  \nabla \Psi .dV+ \int_V \sigma \tau_i \Psi.dV-\int_V\tau_i q .dV= 0.
\eeq 

Now, when we expand $\Psi$ into basis functions, with the basis functions $b_j$ essentially being the same as the trial functions $\tau_i$ (i.e. when $i=j$), as 
$$
\Psi = \sum_{j=0}^{N_{dof}-1} b_j \Psi_j
$$
we arrive at the ``per DOF" version of the finite element equation

\beqn \label{eq:finiteelement2Dtriangle}
 \sum_{j=0}^{N_{dof}-1}
 \biggr[
\int_{\partial V} (\Omega \bigcdot \hat{n}) \tau_i (\tilde{\Psi} - \Psi_j.b_j).dS  
+\int_V \Psi_j \tau_i \Omega \bigcdot  \nabla b_j .dV
+ \int_V \Psi \sigma \tau_i b_j.dV
 \biggr]= \int_V\tau_i q .dV.
\eeqn 
\newline
From this equation we see that we need expressions for the shape function $\tau_i$ and $b_j$ as well as the derivative of the trial function $\nabla \tau_i$. We also need a means to efficiently integrate these functions over surface and volume.

\subsection{Piecewise linear shape functions on a 2D triangle}
For a two dimensional simulation using triangular elements we seek to map a triangle in cartesian space to a reference triangle in natural coordinates. We do this because we can develop a method to perform integration or differentation for the reference triangle that can be mapped to a triangle of any shape and location. An example of the two triangles in different coordinate space is shown in Figure \ref{fig:twodreferenceelement}.

\begin{figure}[H]
\centering
\includegraphics[width=0.8\linewidth]{LatexDraw/TwoD_ReferenceElement}
\caption{Mapping of a 2D triangle to a reference triangle in natural coordinates.}
\label{fig:twodreferenceelement}
\end{figure}

The linear basis functions for the reference triangle are

\beq 
N_0(\xi,\eta) &= 1 - \xi - \eta \\
N_1(\xi,\eta) &= \xi \\
N_2(\xi,\eta) &= \eta. \\
\eeq 
From these functions we can interpolate the point $(x,y)$ with the following

\beq 
x &= N_0 x_0 + N_1 x_1 + N_2 x_2 \\
y &= N_0 y_0 + N_1 y_1 + N_2 y_2 \\
\eeq 
We can now express $x$ and $y$ as functions of $\xi$ and $\eta$ by substituting the expressions for $N_0$, $N_1$ and $N_2$ into the expressions for $x$ and $y$

\beq 
x &= (1-\xi-\eta)x_0 + (\xi)x_1 + (\eta)x_2 \\
&= x_0 -\xi x_0 -\eta x_0 +\xi x_1 +\eta x_2 \\
&= x_0 +(x_1 - x_0)\xi + (x_2 - x_0)\eta
\eeq 
and
\beq 
y &= (1-\xi-\eta)y_0 + (\xi)y_1 + (\eta)y_2 \\
&= y_0 -\xi y_0 -\eta y_0 +\xi y_1 +\eta y_2 \\
&= y_0 +(y_1 - y_0)\xi + (y_2 - y_0)\eta
\eeq 
\newline
In terms of the vectors from vertex $0$ to the other two vertices (refer to Figure \ref{fig:twodreferenceelement}) we can write this as

\beqn \label{eq:x2Dnat}
x = x_0 + \text{v}_{01x} \xi +\text{v}_{02x} \eta
\eeqn 
\beqn \label{eq:y2Dnat}
y = y_0 + \text{v}_{01y} \xi +\text{v}_{02y} \eta
\eeqn 
\newline
which is in the form of a linear transformation and from which we can determine the very important Jacobian matrix

\begingroup
\renewcommand*{\arraystretch}{1.5}
\beqn \label{eq:jacobiantriangle} 
\mathbf{J }= 
\begin{bmatrix}
\dfrac{dx}{d\xi}     & \dfrac{dx}{d\eta} \\
\dfrac{dy}{d\xi}     & \dfrac{dy}{d\eta} \\
\end{bmatrix}=
\begin{bmatrix}
\text{v}_{01x}  & \text{v}_{02x}  \\
\text{v}_{01y}  & \text{v}_{02y}  \\
\end{bmatrix}
=
\begin{bmatrix}
(x_1 - x_0) & (x_2 - x_0)  \\
(y_1 - y_0)  & (y_2 - y_0) \\
\end{bmatrix}.
\eeqn
\endgroup
\newline
The first application of the Jacobian will be for the integration of the trial or basis function in the finite element method. For simplicity let us consider the integration of a function over $x$ and $y$ which can be transformed to an integration of the linear transformation of function f, i.e. function g, over $\xi$ and $\eta$ using fundamental linear algebra. This integration is then

\beq 
\int \int f(x,y) .dx.dy = \int \int g (\xi,\eta).|J|.d\xi.d\eta
\eeq 
\newline
where $|J|$ is the determinant of the Jacobian. This integration can then easily be done either analytically or by using a quadrature rule. For the reference triangle this can easily be done using the method of undetermined coefficients as detailed in appendix \ref{appendix:trianglequadrature}.
\newline
\newline 
We need to define one more item that is related to the finite element method and that is the derivative of the basis functions, $\nabla N_i(\xi,\eta)$, which can be developed by noting that

\beq 
\frac{\partial N_i}{\partial \xi} = 
\frac{\partial N_i}{\partial x}\cdot \frac{\partial x}{\partial \xi} \ + \ 
\frac{\partial N_i}{\partial y}\cdot  \frac{\partial y}{\partial \xi}
\eeq 
\beq 
\frac{\partial N_i}{\partial \eta} = 
\frac{\partial N_i}{\partial x}\cdot \frac{\partial x}{\partial \eta} \ + \ 
\frac{\partial N_i}{\partial y}\cdot  \frac{\partial y}{\partial \eta}
\eeq 
\newline
which can be written as

\begingroup
\renewcommand*{\arraystretch}{1.5}
\beq
\begin{bmatrix}
\dfrac{\partial N_i}{\partial \xi} \\
\dfrac{\partial N_i}{\partial \eta}
\end{bmatrix}
&=
\begin{bmatrix}
\dfrac{\partial x}{\partial \xi}   &\dfrac{\partial y}{\partial \xi} \\
\dfrac{\partial x}{\partial \eta}   &\dfrac{\partial y}{\partial \eta} \\
\end{bmatrix}
\begin{bmatrix}
\dfrac{\partial N_i}{\partial x} \\
\dfrac{\partial N_i}{\partial y}
\end{bmatrix} \\
\therefore 
\begin{bmatrix}
\dfrac{\partial N_i}{\partial \xi} \\
\dfrac{\partial N_i}{\partial \eta}
\end{bmatrix}
&= \mathbf{J}^T 
\begin{bmatrix}
\dfrac{\partial N_i}{\partial x} \\
\dfrac{\partial N_i}{\partial y}
\end{bmatrix}.
\eeq 
\endgroup

Now we can invert $\mathbf{J}^T$ to get

\begingroup
\renewcommand*{\arraystretch}{1.5}
\beqn \label{eq:derivativeNtriangle}
\begin{bmatrix}
\dfrac{\partial N_i}{\partial x} \\
\dfrac{\partial N_i}{\partial y}
\end{bmatrix}
&=
(\mathbf{J}^T)^{-1}
\begin{bmatrix}
\dfrac{\partial N_i}{\partial \xi} \\
\dfrac{\partial N_i}{\partial \eta}
\end{bmatrix}
\eeqn 
\endgroup

and since the inverse of a $2\stimes 2$ matrix is given by

\beq 
\begin{bmatrix}
a & b\\
c & d
\end{bmatrix}^{-1}
=\frac{1}{ad-bc} 
\begin{bmatrix}
d & -b\\
-c & a
\end{bmatrix}
\eeq 

we have

\begingroup
\renewcommand*{\arraystretch}{1.5}
\beqn \label{eq:jacobianinversetriangle}
(\mathbf{J}^T)^{-1} =
\begin{bmatrix}
\dfrac{\partial x}{\partial \xi}   &\dfrac{\partial y}{\partial \xi} \\
\dfrac{\partial x}{\partial \eta}   &\dfrac{\partial y}{\partial \eta} \\
\end{bmatrix}^{-1}
&=
\frac{1}{|J|}
\begin{bmatrix}
\dfrac{\partial y}{\partial \eta}   & -\dfrac{\partial y}{\partial \xi} \\
-\dfrac{\partial x}{\partial \eta}  & \dfrac{\partial x}{\partial \xi} 
\end{bmatrix}
\eeqn
\endgroup


\newpage
\textbf{In summary:}\newline 
In two dimensions using triangular cells, we wish to represent equation \ref{eq:finiteelement2Dtriangle} (repeated here)

\beq
 \sum_{j=0}^{N_{dof}-1}
 \biggr[
\int_{\partial V} (\Omega \bigcdot \hat{n}) \tau_i (\tilde{\Psi} - \Psi_j.b_j).dS  
+\int_V \Psi_j \tau_i \Omega \bigcdot  \nabla b_j .dV
+ \int_V \Psi \sigma \tau_i b_j.dV
 \biggr]= \int_V\tau_i q .dV.
\eeq
\newline
where $\tau_i(\xi,\eta) = b_i(\xi,\eta) = N_i(\xi,\eta)$, the latter being the shape functions for each of the vertices of a triangular cell. We can evaluate these shape functions at any value of $\xi$ and $\eta$ within the reference triangle shown in Figure \ref{fig:twodreferenceelement} as

\beq 
N_0(\xi,\eta) &= 1 - \xi - \eta \\
N_1(\xi,\eta) &= \xi \\
N_2(\xi,\eta) &= \eta. \\
\eeq 
\newline
The derivatives of the shape functions $\nabla \tau_i(\xi,\eta) = \nabla b_i(\xi,\eta) = \nabla N_i(\xi,\eta)$ can be computed with a given triangle using equations \ref{eq:derivativeNtriangle}, \ref{eq:jacobianinversetriangle} and \ref{eq:jacobiantriangle}, repeated respectively here

\begingroup
\renewcommand*{\arraystretch}{1.5}
\beq
\begin{bmatrix}
\dfrac{\partial N_i}{\partial x} \\
\dfrac{\partial N_i}{\partial y}
\end{bmatrix}
&=
(\mathbf{J}^T)^{-1}
\begin{bmatrix}
\dfrac{\partial N_i}{\partial \xi} \\
\dfrac{\partial N_i}{\partial \eta}
\end{bmatrix}
\eeq 
\endgroup

\begingroup
\renewcommand*{\arraystretch}{1.5}
\beq
(\mathbf{J}^T)^{-1} =
\begin{bmatrix}
\dfrac{\partial x}{\partial \xi}   &\dfrac{\partial y}{\partial \xi} \\
\dfrac{\partial x}{\partial \eta}   &\dfrac{\partial y}{\partial \eta} \\
\end{bmatrix}^{-1}
&=
\frac{1}{|J|}
\begin{bmatrix}
\dfrac{\partial y}{\partial \eta}   & -\dfrac{\partial y}{\partial \xi} \\
-\dfrac{\partial x}{\partial \eta}  & \dfrac{\partial x}{\partial \xi} 
\end{bmatrix}
\eeq
\endgroup

\begingroup
\renewcommand*{\arraystretch}{1.5}
\beq
\mathbf{J }= 
\begin{bmatrix}
\dfrac{dx}{d\xi}     & \dfrac{dx}{d\eta} \\
\dfrac{dy}{d\xi}     & \dfrac{dy}{d\eta} \\
\end{bmatrix}=
\begin{bmatrix}
\text{v}_{01x}  & \text{v}_{02x}  \\
\text{v}_{01y}  & \text{v}_{02y}  \\
\end{bmatrix}
=
\begin{bmatrix}
(x_1 - x_0) & (x_2 - x_0)  \\
(y_1 - y_0)  & (y_2 - y_0) \\
\end{bmatrix}.
\eeq
\endgroup

Integrals of the form 
\beq 
&\int_V \tau_i b_j .dV   \quad \quad \text{and } &&\int_V \tau_i \Omega \bigcdot \nabla b_j .dV
   \quad \quad \text{and } && \int_V \tau_i .dV
\eeq 
can now be evaluated using a quadrature rule of the form
\beq 
\int_V f(x,y) .dx.dy = \int_{-1}^1 \int_0^{1-\eta} g(\xi,\eta).|J|.d\xi.d\eta = |J| \sum_{q=0}^{N_q-1} w_q g(\xi_q,\eta_q)
\eeq 
where the quadrature weights and points are given in appendix \ref{appendix:trianglequadrature} and $g(\xi,\eta)$ are combinations of the basis functions $N_i$. Note the dot-product where the derivative of the base function is used. Also note that the derivative of the basis function results in a vector.




\newpage 
\subsection{Piecewise linear shape functions on a 2D polygon}
The development of the methodology as applied to a 2D triangle has direct application when applied to a polygon since a polygon most likely presents more complexity than classical reference elements like quadrilaterals and therefore using triangles as grp_subsets of polygons overcomes this complexity. The use of subset triangles for the representation of a polygon was presented by Bailey \& Adams in \cite{BaileyAdamsPWLPolygons} the same authors which subsequently studied bi-linear basis functions \cite{BaileyAdamsPWBLPolygons}. From the latter paper it is this author's judgement that bi-linear basis functions offer little benefit over their linear counterparts and we will therefore pursue the linear methods presented in \cite{BaileyAdamsPWLPolygons}. The basis functions for each vertex of a polygon are of the form

\beqn 
P_i(x,y) = N_i(x,y) + \beta_i N_c(x,y)
\eeqn 
\newline
where the functions $N_i$ and $N_c$ are the standard linear functions defined on triangles. The subscripts $i$ and $c$ refer to the vertices $i$ and center of the polygon, respectively. The $\beta_i$ value is a weighting constant defined such that 

\beqn 
\begin{bmatrix}
x \\ y
\end{bmatrix}_{c}
= \sum_{s=0}^{N_{s}} \beta_s 
\begin{bmatrix}
x \\ y
\end{bmatrix}_{s,avg}.
\eeqn
\newline
Naturally it follows that $\beta_s = \dfrac{1}{N_{s}}$ where $N_s$ is the amount of sides. $[x \ y]_{s,avg}$ is the average coordinate of the two vertices of a side. An example basis function is shown in Figure \ref{fig:twodpolygon}. It should be noted that a single basis function now requires integration on each of the sub-triangles of the polygon instead of just a single one.

\begin{figure}[H]
\centering
\includegraphics[width=0.5\linewidth]{LatexDraw/TwoD_Polygon}
\caption{Basis function on a 2D polygon.}
\label{fig:twodpolygon}
\end{figure}


It is hard to visualize that this approach leads to an equivalent representation but with a few tests one can establish that this is equivalent. Some aspects of the paper presenting this method \cite{BaileyAdamsPWLPolygons} that are not intuitive is the explanation that integration is now over all sides. Also, the paper does not explicitly state that the cell centroid never features as an unknown in the solution. The customary assembly of the matrix in triangle or quadrilateral based meshes is to assemble cell-by-cell with an inner loop over the DOF of the cell. This approach is essentially the same but the paper states that integration is per side without saying that each DOF (except the cell center) of each side is also an inner loop of this integration. The subtle differences in the two algorithms can be seen in algorithm \ref{algo:triangle} and \ref{algo:polygon}.
\newline
\newline
The method is versatile enough to applied to triangles and quadrilaterals where examples of the shape functions are shown in Figure

\begin{figure}[H]
\centering
\includegraphics[width=1.0\linewidth]{LatexDraw/TwoD_PolygonTriQuad}
\caption{Basis functions on a triangle and on a quadrilateral.}
\label{fig:twodpolygontriquad}
\end{figure}


\vspace{0.5cm}
\begin{algorithm}[H]
\ForEach{cell}{
$A^{N_v\stimes N_v}$ $\longleftarrow$ Initialize matrix \;
$b^{N_v\stimes 1}$ $\longleftarrow$ Initialize right hand side \;
\ForEach{vertex $i$}{
\ForEach{vertex $j$}{
\ForEach{quadrature point qp}{
$a_{ij}=a_{ij} + $ $\longleftarrow$ Eq. with $\tau_i$ and $b_j$ integrated over cell\;
$b_i = b_i +$ $\longleftarrow$ Eq. with $\tau_i$ and $b_j$ integrated over cell\;
}
}
}
$\phi$ $\longleftarrow$ Solve system\;
}
\caption{Triangle Based algorithm\label{algo:triangle}}
\end{algorithm}

\vspace{0.5cm}
\begin{algorithm}[H]
\ForEach{cell}{
$A^{N_v\stimes N_v}$ $\longleftarrow$ Initialize matrix \;
$b^{N_v\stimes 1}$ $\longleftarrow$ Initialize right hand side \;
\ForEach{vertex $i$}{
\ForEach{vertex $j$}{
\ForEach{sub-triangle s}{
\ForEach{quadrature point qp}{
$a_{ij}=a_{ij} + $ $\longleftarrow$ Eq. with $\tau_i$ and $b_j$ integrated over sub-triangle\;
$b_i = b_i +$ $\longleftarrow$ Eq. with $\tau_i$ and $b_j$ integrated over sub-triangle\;
}
}
}
}
$\phi$ $\longleftarrow$ Solve system\;
}
\caption{Polygon Based algorithm\label{algo:polygon}}
\end{algorithm}




\newpage
\subsection{Piecewise linear shape functions on a 3D tetrahedron}
For a three dimensional simulation using tetrahedral elements we seek to map a tetrahedron in cartesian space to a reference tetrahedron in natural coordinates. We do this because we can develop a method to perform integration or differentation for the reference tetrahedron that can be mapped to a tetrahedron of any shape and location. An example of the two tetrahedron in different coordinate space is shown in Figure \ref{fig:threedreferenceelement}.

\begin{figure}[H]
\centering
\includegraphics[width=1\linewidth]{LatexDraw/ThreeD_ReferenceElement}
\caption{Mapping of a 3D tetrahedron to a reference tetrahedron in natural coordinates.}
\label{fig:threedreferenceelement}
\end{figure}

The linear basis functions for the reference tetrahedron are

\beq 
N_0(\xi,\eta,\zeta) &= 1 - \xi - \eta - \zeta \\
N_1(\xi,\eta,\zeta) &= \xi \\
N_2(\xi,\eta,\zeta) &= \eta \\
N_3(\xi,\eta,\zeta) &= \zeta \\
\eeq 

From these functions we can interpolate the point $(x,y,z)$ with the following

\beq 
x &= N_0 x_0 + N_1 x_1 + N_2 x_2 + N_3 x_3 \\
y &= N_0 y_0 + N_1 y_1 + N_2 y_2 + N_3 y_3 \\
z &= N_0 z_0 + N_1 z_1 + N_2 z_2 + N_3 z_3 \\
\eeq 
\newline
We can express $x$, $y$ and $z$ as functions of $\xi$, $\eta$ and $\zeta$ by substituting the basis functions into these expression to obtain

\beq 
x &= x_0 + (x_1-x_0)\xi + (x_2-x_0)\eta + (x_3-x_0)\zeta \\
y &= y_0 + (y_1-y_0)\xi + (y_2-y_0)\eta + (y_3-y_0)\zeta \\
z &= z_0 + (z_1-z_0)\xi + (z_2-z_0)\eta + (z_3-z_0)\zeta \\
\eeq 
\newline
In terms of the vectors from vertex $0$ to the other three vertices (refer to Figure \ref{fig:threedreferenceelement}) we can write this as

\beqn 
x &= x_0 + v_{01x}\xi + v_{02x}\eta + v_{03x}\zeta \\
\eeqn 
\beqn 
y &= y_0 + v_{01y}\xi + v_{02y}\eta + v_{03y}\zeta \\
\eeqn 
\beqn 
z &= z_0 + v_{01z}\xi + v_{02z}\eta + v_{03z}\zeta \\
\eeqn 
\newline
which is in the form of linear transformation and from which we can determine the very important Jacobian matrix

\begingroup
\renewcommand*{\arraystretch}{1.5}
\beqn \label{eq:jacobiantetrahedron} 
\mathbf{J }= 
\begin{bmatrix}
\dfrac{dx}{d\xi}     & \dfrac{dx}{d\eta}  &  \dfrac{dx}{d\zeta} \\
\dfrac{dy}{d\xi}     & \dfrac{dy}{d\eta}  &  \dfrac{dy}{d\zeta} \\
\dfrac{dz}{d\xi}     & \dfrac{dz}{d\eta}  &  \dfrac{dz}{d\zeta} \\
\end{bmatrix}=
\begin{bmatrix}
\text{v}_{01x}  & \text{v}_{02x} & \text{v}_{03x}  \\
\text{v}_{01y}  & \text{v}_{02y} & \text{v}_{03y}  \\
\text{v}_{01z}  & \text{v}_{02z} & \text{v}_{03z}  \\
\end{bmatrix}
=
\begin{bmatrix}
(x_1 - x_0) & (x_2 - x_0) & (x_3-x_0) \\
(y_1 - y_0)  & (y_2 - y_0) & (y_3-y_0) \\
(z_1 - z_0)  & (z_2 - z_0) & (z_3-z_0) \\
\end{bmatrix}.
\eeqn
\endgroup
\newline

As was the case with the triangle we now seek

\begingroup
\renewcommand*{\arraystretch}{1.5}
\beqn \label{eq:derivativeNtetrahedron}
\begin{bmatrix}
\dfrac{\partial N_i}{\partial x} \\
\dfrac{\partial N_i}{\partial y} \\
\dfrac{\partial N_i}{\partial z}
\end{bmatrix}
&=
(\mathbf{J}^T)^{-1}
\begin{bmatrix}
\dfrac{\partial N_i}{\partial \xi} \\
\dfrac{\partial N_i}{\partial \eta} \\
\dfrac{\partial N_i}{\partial \zeta} \\
\end{bmatrix}
\eeqn 
\endgroup
where we need to find the inverse of transpose of the Jacobian, with the transpose given by
\begingroup
\renewcommand*{\arraystretch}{1.5}
\beqn \label{eq:jacobiantransposetetrahedron} 
\mathbf{J }^T=
\begin{bmatrix}
\text{v}_{01x}  & \text{v}_{01y} & \text{v}_{01z}  \\
\text{v}_{02x}  & \text{v}_{02y} & \text{v}_{02z}  \\
\text{v}_{03x}  & \text{v}_{03y} & \text{v}_{03z}  \\
\end{bmatrix}
\eeqn
\endgroup
we can compute the determinant $|J^T|=|J|$  as
\beqn 
|J| = |J^T| =& \text{v}_{01x}       (\text{v}_{02y} \text{v}_{03z}    - \text{v}_{03y}  \text{v}_{02z} ) \\
     - &\text{v}_{01y}       (\text{v}_{02x} \text{v}_{03z}    - \text{v}_{03x}  \text{v}_{02z} ) \\
     + &\text{v}_{01z}       (\text{v}_{02x} \text{v}_{03y}    - \text{v}_{03x}  \text{v}_{02y} )
\eeqn

we now compute the matrix of minors and subsequently the matrix of cofactors
\begingroup
\renewcommand*{\arraystretch}{1.5}
\beq
\begin{bmatrix}
m_{11} &m_{12} &m_{13} \\
m_{21} &m_{22} &m_{23} \\
m_{31} &m_{32} &m_{33} \\
\end{bmatrix}
\eeq
\endgroup
where
\beq 
m_{11} &= \text{v}_{02y}  \text{v}_{03z}       -        \text{v}_{03y}\text{v}_{02z} \\
m_{12} &= \text{v}_{03x}\text{v}_{02z}         -        \text{v}_{02x}  \text{v}_{03z} \\
m_{13} &= \text{v}_{02x}  \text{v}_{03y}       -        \text{v}_{03x}\text{v}_{02y} \\
m_{21} &=  \text{v}_{03y}\text{v}_{01z}         -        \text{v}_{01y}  \text{v}_{03z} \\
m_{22} &= \text{v}_{01x}  \text{v}_{03z}       -        \text{v}_{03x}\text{v}_{01z} \\
m_{23} &= \text{v}_{03x}\text{v}_{01y}         -        \text{v}_{01x}  \text{v}_{03y}\\
m_{31} &= \text{v}_{01y}  \text{v}_{02z}        -        \text{v}_{02y}\text{v}_{01z} \\
m_{32} &= \text{v}_{02x}\text{v}_{01z}          -       \text{v}_{01x}  \text{v}_{02z}\\
m_{33} &= \text{v}_{01x}  \text{v}_{02y}       -        \text{v}_{02x}\text{v}_{01y}
\eeq 
\newline
To see how this was done a good explanation is availabe in \cite{MathisFunMatrixInverse}. We finally transpose this matrix and find the inverse of the transpose of the Jacobian as

\begingroup
\renewcommand*{\arraystretch}{1.5}
\beqn \label{eq:jacobiantransposeinversetetrahedron} 
(\mathbf{J }^T)^{-1}=
\begin{bmatrix}
\dfrac{dx}{d\xi}     & \dfrac{dy}{d\xi}  &  \dfrac{dz}{d\xi} \\
\dfrac{dx}{d\eta}     & \dfrac{dy}{d\eta}  &  \dfrac{dz}{d\eta} \\
\dfrac{dx}{d\zeta}     & \dfrac{dy}{d\zeta}  &  \dfrac{dz}{d\zeta} \\
\end{bmatrix}
=
\frac{1}{|J|}
\begin{bmatrix}
m_{11} &m_{21} &m_{31} \\
m_{12} &m_{22} &m_{32} \\
m_{13} &m_{23} &m_{33} \\
\end{bmatrix}
\eeqn
\endgroup

Applying this process to the finite element problem is similar to procedure for the 2D triangle quadrature points provided in appendix \ref{appendix:tetrahedronquadrature}.


\newpage
\subsection{Piecewise linear shape functions on a 3D Polyhedron}
From the same authors that presented the shape functions for polygons (in \cite{BaileyAdamsPWLPolygons}) another paper was presented for polyhedrons in \cite{BaileyAdamsPolyhedral} where the shape function for each vertex $i$ of the polyhedron is given by

\beqn \label{eq:shapefunctionpolyhedron}
P_i(x,y,z) = N_i (x,y,z) +\sum^{faces \ at \ i}  \beta_f N_f(x,y,z) + \alpha_c N_c(x,y,z)
\eeqn 
\newline
where the functions $N(x,y,z)$ are the standard linear shape functions defined on a tetrahedron. $\beta_s$ and $\alpha_c$  is the weight that gives the face midpoint, $\bar{r}_{fc}$, and cell mid-point, $\bar{r}_{cc}$, rsepectively from the sum of the vertices that constitute them. i.e.

\beqn 
\bar{r}_{fc} =
\begin{bmatrix}
x \\ y \\ z
\end{bmatrix}_{fc}
= \sum_{v=0}^{N_{vf}} \beta_f
\begin{bmatrix}
x \\ y \\z
\end{bmatrix}_{v}.
\eeqn

\beqn 
\bar{r}_{cc} =
\begin{bmatrix}
x \\ y \\ z
\end{bmatrix}_{cc}
= \sum_{v=0}^{N_{vc}} \alpha_c
\begin{bmatrix}
x \\ y \\z
\end{bmatrix}_{v}.
\eeqn
\newline
where $N_{vf}$ is the number of vertices for the given face and $N_{vc}$ is the number of vertices for the entire cell. Naturally it follows that $\beta_f = \frac{1}{N_{vf}}$ and $\alpha_c = \frac{1}{N_{vc}}$. The format of equation \ref{eq:shapefunctionpolyhedron} is not intuitive at first sight ... it is hard to comprehend the summation over faces ``at j", but let us try to clarify this with a diagram (see Figure \ref{fig:threedtetrahedral}).

\begin{figure}[H]
\centering
\includegraphics[width=1\linewidth]{LatexDraw/ThreeDTetrahedral}
\caption{Connection of the vertex of interest to the tetrahedrons that comprise the cell.}
\label{fig:threedtetrahedral}
\end{figure}

Firstly we split the polyhedron into faces, where each face can be a polygonal face. Each face is then split into a number of sides. A side is a tetrahedron, corresponding to a face, which is formed from each edge of the polygonal face where the vertex collection are the two vertices of the edge, the face center and the cell center. In other words the face center and the two vertices of the edge forms a triangle, and the cell center makes it a tetrahedron.
\newline 
\newline
As was the case with the polygon, all the other vertices $j$ (not $i$) are connected to the vertex of interest $i$ through the cell center. Again, we don't include the cell center as a point in the simulation so we have to spread its effect through to each of the vertices using the $\alpha_c$ factor. We also have face centered shape functions associated with the division of each polygonal face into sides. On each face, to which vertex $i$ belongs, the shape functions defined on the face centers will protrude into the tetrahedron under consideration (i.e. the tetrahedron associated with vertex $i$ associated with face $f$, side $s$). Therefore more clearly we can express the shape functions on a tetrahedron-by-tetrahedron basis

\beqn
P_i^{tet}(x,y,z) = 
\begin{cases}
\alpha_c N_c (x,y,z)    \quad \quad &\text{ no matter which tetrahedron} \\
+\beta_f N_f(x,y,z)  \quad \quad &\text{ if vertex }i \text{ is part of the face} \\
+N_i (x,y,z) \quad \quad &\text{ if vertex }i \text{ is part of the face-side pair} \\
\end{cases}
\eeqn
\newline
Figure \ref{fig:threedtetrahedralshape} shows the influence of a shape function (centered on a specific point as denoted by the start of an arrow) from a specific vertex (color). The orange colored vertex's influence is shown on the left most figure where the shape function is then the full equation because all of the conditions are met; i.e. $\alpha_c N_c$ is always present, vertex $i$ is indeed part of the face where this tetrahedron is defined and therefore $\beta_f N_f$ is present, finally it is also part of the side of the tetrahedron and therefore its basic shape function $N_i$ is present.
\newline
\newline
The middle figure shows the red vertex as a vertex that only has the contribution of $\alpha_c N_c$ and $\beta_f N_f$ because the red vertex is not on the side comprosing the tetrahedron of interest. The rightmost figure shows only the contribution of $\alpha_c N_c$ because the blue vertex is not part of any adjacent face. 

\begin{figure}[H]
\centering
\includegraphics[width=1\linewidth]{LatexDraw/ThreeDTetrahedralShape}
\caption{Influence of different vertices on the shape functions of within a tetrahedral portion of the cell.}
\label{fig:threedtetrahedralshape}
\end{figure}

\vspace{0.5cm}
\begin{algorithm}[H]
\ForEach{cell}{
$A^{N_v\stimes N_v}$ $\longleftarrow$ Initialize matrix \;
$b^{N_v\stimes 1}$ $\longleftarrow$ Initialize right hand side \;
\ForEach{vertex $i$}{
\ForEach{vertex $j$}{
\ForEach{face f}{
\ForEach{sub-tetrahedral s}{
\ForEach{quadrature point qp}{
$a_{ij}=a_{ij} + $ $\longleftarrow$ Eq. with $\tau_i$ and $b_j$ integrated over sub-triangle\;
$b_i = b_i +$ $\longleftarrow$ Eq. with $\tau_i$ and $b_j$ integrated over sub-triangle\;
}
}
}
}
}
$\phi$ $\longleftarrow$ Solve system\;
}
\caption{Polyhedron Based algorithm\label{algo:polyhedron}}
\end{algorithm}






\newpage
\chead{Code design}
\section{Code design}
\subsection{Overall strategy}
The neutron transport equation, with the scattering term and angular flux expanded, results in a large amount of variables to be stored and computed. Before introducing any form of spatial discretization the code has to handle the scalar flux moments $\phi_0 ... \phi_L$ for which we have a set for each group. When a spatial discretization scheme, along with a collection of cells, is introduced the problem size is scaled by the product of the number of cells and each cells respective degrees of freedom (DOF). Therefore the primary strategy of the code design is to support domain decomposition.

\subsection{Handling the mesh}
All mesh related items will be contained within the \xmloption{namespace} called \xmltag{chi\_mesh}. The primary item will be a storage space that will allow us to store all mesh entities into a global container we call the \xmltag{chi\_mesh::MeshHandler} class. All mesh creation and manipulations happen with a mesh handler, set to``current", and therefore in theory the code supports the use of multiple meshes, but that is a design feature to explore later.
\newline
\newline
Mesh construction, in general, can be split into 3 different methods:
\begin{itemize}
\item Use of pre-generated meshes (PGM). Such meshes have been built outside of the code and is in a form conducive to the direct construction of internal mesh data structures. This is a rather complex method since it requires knowledge of the source of such methods.
\item Generation of meshes within the code, i.e. internally generated meshes (IGM). Such a method will require the complete specification of the geometry within the code input structure as well as the algorithms required to create fully functional meshes with boundary identification.
\item Partial internally generated meshing (PIGM) . This method mostly implies the extrusion of a 2D surfaces to a 3D mesh but could extend to a 3D-printed type mesh..
\end{itemize}


\subsubsection{Pre-generated meshes}
Pre-generated meshes (PGMs) is a large topic for computational physics and not just a topic in this work. Many well developed packages exist that can produce high quality 3D meshes very efficiently and in most cases in parallel. Historically a software package generates a mesh file that is to be read by the code that uses it and for most applications a single file with all the mesh information suffices. However, in consideration of scaling simulations (i.t.o. number of spatial unknowns) one has to consider the ``friendliness" of such meshes towards scaling. 

\newpage
In the simplest case consider a mesh comprising hexahedrals, each cell will at minimum require:

\begin{itemize}[noitemsep]
\item \xmltag{4byte} The cell memory pointer.
\item \xmltag{byte} Number of vertices.
\item \xmltag{4byte} Pointer to vertex array.
\item \xmltag{8$\stimes$4bytes} Vertex index array.
\item \xmltag{byte} Number of faces.
\item \xmltag{4byte} Pointer to face inforation array.
\item \xmltag{4$\stimes$4 bytes} for each face to store vertex indices, \xmltag{96 bytes} total over the 6 faces.
\item \xmltag{4bytes} for each face storing the neighboring cell or boundary, \xmltag{16bytes} total over the 6 faces.
\item \xmltag{4bytes} The material index.
\end{itemize}

This is a total of 162 bytes for the storage of each cell and therefore we can fit approximately 6.6 million cells into 1 GB of memory. In practice though cell information includes a number of other variables that ease the lookup of information and the calculation of simulation inputs and in this regard it is more realistic to assume less than this number. However, if each process in a parallel simulation had to read and process the pregenerated mesh file then the total size of the mesh will essentially be capped by the average memory per core which is currently only as high as 3.5 GB per core (i.e. the Lawrence Livermore National Laboratory's Quartz cluster).
\newline
\newline
Scalable PGMs therefore require support for domain decomposition and methods to do so are being developed. The concept of generating cell counts in excess of 1 billion with a mesh generator is also a cause for concern since such a meshing tool will also require a means to do so efficiently and mostly in parallel as well. The end-specification however remains constant: PGMs need to exist in multiple files that are conducive to each process reading only the mesh that is allocated to it. Such efforts will require advanced domain decomposition and load balancing.
\newline
\newline
Support for PGMs has been an engineered features of ChiTech from the start because of the light-weight structure employed for containing mesh information. Interfaces can easily be developed to move meshes of any format into the database. To this end mesh information is stored within ChiTech as either \xmltag{CellPolygon} or \xmltag{CellPolyhedron}. The data structures for each of these is simple:
\vspace{0.5cm}
\begin{lstlisting}[language=c++]
class chi_mesh::Cell
{
public:
  int cell_global_id;
  int cell_local_id;
  std::pair<int,int> xy_partition_indices;
  std::tuple<int,int,int> xyz_partition_indices;
  int partition_id;
  Vertex centroid;
  int material_id;
 }
\end{lstlisting}
\newpage
\begin{lstlisting}[language=c++]
class chi_mesh::CellPolygon : public chi_mesh::Cell
{
public:
  std::vector<int>  v_indices;
  std::vector<int*> edges; ///< Stores arrays of edge indices
  std::vector<chi_mesh::Vector> edgenormals;
};
\end{lstlisting}
\vspace{0.25cm}
\begin{lstlisting}[language=c++]
class chi_mesh::CellPolyhedron : public chi_mesh::Cell
{
public:
  std::vector<int>       v_indices;
  std::vector<PolyFace*> faces;
};

struct chi_mesh::PolyFace
{
  std::vector<int> v_indices;
  std::vector<int*> edges;
  int face_indices[3];

  chi_mesh::Normal geometric_normal;
  chi_mesh::Vertex face_centroid;
};
\end{lstlisting}

In these data structures the normals and centroids can be computed as they are loaded and knowledge of neighbors is incorporated via \xmltag{edges} or \xmltag{face\_indices}. The base-class \xmltag{Cell} also doubles as a very lightweight ``ghost" cell which can conveniently be communicated between process to ease algorithm detail downstream of mesh loading. This is brings one to the topic of global cell indices which ChiTech uses. This eases the burden on engineering programmers where the concepts of graphs are not always solid. It also assists in restarting simulations with different partitioning strategies, sometimes even after a field function has already been computed. This strategy has proved very successful.
\newline
\newline
Reading decomposed 2D meshes will be incorporated in the \xmltag{VolumeMesherPredefined2D} volume mesher, and reading decomposed 3D meshes will be incorporated through the \xmltag{VolumeMesherPredefined3D} volume meshers.


\newpage 
\subsubsection{Internally generated meshes}
In the ideal case the code would have had the capability of generating, in parallel, a fully functional 3D mesh similar to what can be read with the predefined 3D mesh reader, however, a simple method of obtaining 3D meshes is to extrude 2D meshes and if the 2D meshes themselves are generated with the code then technically such a method is considered completely internally generated. The planning therefore is to use a Delaunay triangulation mesher to generate a 2D mesh and to extrude this mesh. The triangulation mesher will be named \xmltag{SurfaceMesherDelaunay} and the corresponding extruder will be named \xmltag{VolumeMesherExtruder}.

\subsubsection{Partially internally generated meshes}
As a subset of a 3D mesh generator using extrusion is the case where the 2D surface mesh is pre-defined and for this case we will name this process the \xmltag{SurfaceMesherPredefined} mesher which will feed into the \xmltag{VolumeMesherExtruder}.

\subsubsection{Meshing utilities}
In support of meshing operations a number of data structures and classes are defined to make the namespace all-inclusive. These are:
\begin{itemize}[noitemsep]
\item \xmltag{Vector}. Also doubling for \xmltag{Vertex}, \xmltag{Normal} and \xmltag{Node}. A general data structure that stores a vector of 3 components ($x,y,z$) and supports vector arithmetic, dot-products, cross-products and the L2-norm.
\item \xmltag{Edge} and \xmltag{EdgeLoop} for the storage of edges.
\item \xmltag{Face} and \xmltag{Polyface} for the storage of triangular and polygonal faces, respectively.
\item \xmltag{LineMesh} for storing a collection of lines used for the Delaunay mesher and to define boundaries.
\item \xmltag{SurfaceMesh} for storing either surface meshes read from file or store surface meshes generated within the code.
\item \xmltag{Boundary}, \xmltag{Region} and \xmltag{MeshContinuum} with each nesting from left to right is used to define continuum on which the mesh is defined.
\item \xmltag{CellTriangle}, \xmltag{CellPolygon} and \xmltag{CellPolyhedron} as a basic data structure to be handled by any process downstream. The end-goal of meshing is to have a collection of these.
\item \xmltag{Cellset} a collection of cell id's belonging to the local process.
\end{itemize}



\newpage 
\subsubsection{Mesh partitioning}
The current design for mesh partitioning requires unique global cell indices that map down to local cell indices. Cells are stored in a data structure named \xmltag{MeshContinuum} which gets pushed onto a \xmltag{Region} datastructure during the execution of the volume-mesher. This allows us to facilitate different regions of meshing. Within the \xmltag{MeshContinuum} the cells and their associated nodes are stored along with an important mapping vector \xmltag{local\_cell\_glob\_indices} which contains the global index mapping of local cells. Programmers can simply supply the local cell index to obtain the global cell index (the inverse mapping is also computed by the meshing routines).
\newline
\newline
Only local cells are fully defined, as either \xmltag{CellPolygon} or \xmltag{CellPolyhedron}, whereas non-local cells are stored as a near empty reference to the parent cell type \xmltag{Cell} which only stores its centroid and its partition id. We can call the parent cell type a ``ghost"-cell because it gets used as a placeholder when a cell is not local. Using these ghost cells is very advantageous for communicating data and is very easy to implement with an extruder mesher. An example of this is shown in Figure \ref{fig:cellmapping} below.


\begin{figure}[H]
\centering
\includegraphics[width=0.9\linewidth]{LatexDraw/CellMapping}
\caption{Cell mapping logic.}
\label{fig:cellmapping}
\end{figure}









\newpage
\subsection{Groups and Angles}
As is often the case with neutral particle transport, the notion of a \xmltag{Group} arises mostly because the material properties are dependent on particle energy and hence require discretization in the energy domain. Particles can be transfered (i.e. scattered) back and forth between energy groups and hence we will require a material property called the \xmltag{ScatteringMatrix} which holds the likelihood of a particle scattering from group $g'$ to group $g$. Hence, for $G$ groups, the scattering matrix can be a $G\stimes G$ dense matrix, however in reality we can introduce some sparsity to these matrices since particles can scatter in a band of groups.\newline 

Because the problem size essentially scales with the number of groups, a very useful memory-saving technique is to split the collection of groups into a number of distinct collections, each called a \xmltag{Groupset} with unique properties. One important property that a groupset can have is it \xmltag{ProductQuadrature} rule for integration over angle space. Particular quadrature rules are discussed in the appendices. Discrete ordinate angles directly related to the abscissae used in the quadrature rule and therefore the \xmltag{Angles} are determined by these rules. As will be seen later, angles, like groups, can have similar properties which we can group into \xmltag{AngleSets} in order to speed up parallel computations.
\newline
\newline
An additional feature incorporated into groupsets is \xmltag{Groupset::Subsets} which allows further seperation of anglesets. This has proven advantageous for the development of ``feedstock" to the sweeping algorithm where the parallel efficiency is dominated by the ratio of idle tasks to total number of tasks.


\vspace{1cm}
\subsection{Sweep Plane Data Structure (SPDS)}
When sweeping along a certain direction, cell dependencies are defined by the notion of upstream dependencies. That is, for current cell $i$ ... if the dot product of the sweep direction vector with the surface normal is negative (i.e. $\Omega \bullet \hat{n} < 0$), and the face associated with that surface is connected to another cell $j$ and not a boundary, then cell $i$ is dependent on cell $j$. Connecting all the cells in this fashion for a particular direction results in the computer science concept called a Directed-Graph, and a useful algorithm developed in this field is the \textbf{topological sort algorithm}. A topologically sorted Directed-Graph provides a sorted list of indices which ensures that a seqeuntial reference to the cells pointed by these indices will reference cells with their dependencies already met during each subsequent downstream reference.
\newline
\newline
The \xmltag{boost::graph} library has this particular functionality with the restriction that the graph must be a Directed-Acyclic-Graph (DAG) which essentially means that cells cannot have cyclic dependencies. Fortunately by enforcing the use of strictly convex cells, acyclic dependencies are ensured. This also means that partitioning cuts need to cut through a cell instead of seperating along faces. We must also note that for a parallel implementation we will have a ``fine-view" of the sweep ordering as well as a ``course-view" which is the processor dependency for the given sweep ordering.
\newpage
Sweeping on meshes is handled by the \xmltag{sweep_management} namespace. Given a grid and a direction ($\Omega$), the function call \xmltag{sweep_management::CreateSweepOrdering} will firstly assemble a DAG of the local cells and create a single Sweep Plane Local Subgrid (\xmltag{SPLS}) that contains the indices of all the local cells, sorted topologically against the direction. This provides the ``fine-view" of sweep ordering. This is stored in the vector \xmltag{SPLS.item\_id} on each sweep ordering. Note that at the time of writing only a single sweep plane is assembled for the local sweep, this is because the local sweeping is inherently serial and there is no parallel benefit to using multiple sweep planes, however, future support for memory optimizing schemes are supported by allowing multiple sweep planes. During the  process of generating a local sweep plane, a set (\xmltag{std::set}) of location dependencies is also constructed which feeds into the subsequent step which is building the ``course view" of the SPDS. The dependencies are split into \xmltag{location\_predecessors} and \xmltag{location\_successors} for later use. The latter step will generate one or more Globals Sweep Plane Orderings (\xmltag{GSPO}s) from which the parallel nature arises, with each global sweep plane denoting the locations that are ``in-play". The purpose of the ``course view" sweep planes is hard to explain without further context, however, such a discussion will have to be deferred to later but for now it will be said that these planes are used in the Depth-Of-Graph sweep scheduling algorithm. The final piece of each SPDS is the Flux Data Structure (\xmltag{FLUDS}) which is the topic of the next subsection but required to be mentioned here to give context to Figure \ref{fig:sweepordering} below, which depicts the hierarchy of datastructures.
\vspace{0.5cm}
\begin{figure}[H]
\centering
\includegraphics[width=0.9\linewidth]{LatexDraw/SweepOrdering}
\caption{Sweep ordering structure.}
\label{fig:sweepordering}
\end{figure}






\newpage 
\subsection{Flux Data Structure (FLUDS)}
A flux datastructure needs to cater for a host of requirements including the prevention of having to filter for stored information. PDT's STAPL library operates on the principle of filtering and plans are in progress \cite{AdamsSPDS} to develop a better implementation. Based on this work ChiTech has been equipped with a similar FLUDS.
\newline
\newline
Let us start with the local subgrid traversal (SPLS). A topological sorting of the Directed Graph was constructed during the development of the \xmltag{SweepOrdering} datastructure and contains the order in which local cells are to be traversed in a sweep for a given angleset. One of the first memory saving principles is that a downstream cell can reuse the memory space where its incoming information was stored. We shall denote this concept as a ``lockbox" (\xmltag{vector<pair<int,short>> lockbox}) and to this end one can imagine an operator that places information in the next available slot and opens up a used slot whenever information is withdrawn. With this paradigm the maximum lockbox size is ultimately determined only by the maximum cells in play. It operates as follows (see Figure \ref{fig:lockbox}):

\begin{figure}[H]
\centering
\includegraphics[width=0.9\linewidth]{LatexDraw/Lockbox}
\caption{Graphical depiction of the ``lockbox" concept.}
\label{fig:lockbox}
\end{figure}

Note in the figure above we only need to know the slot index of where a cell-face pair will store its data. The actual locations in the \xmltag{local\_psi} vector can be computed when the angle and group is provided.
\newline

Some discussion is also warranted on an idea developed in \cite{AdamsSPDS}. The idea is that, according to the above paradigm, the local $\Psi$ vector will be sized according to the number of DOFs on the face with the maximum number of DOFs. This has the potential of allocating a large amount of unused memory if, for arguments sake, 95\% of the cells have the same amount of face dofs (say 4) and only 5\% have lets say 12 DOFs. Effectively 95\% of the faces will then allocate space for 12 DOFs per face where it only requires 4. There is a strategy to overcome this but for now this strategy hasn't been employed since we found it hard to test.


\vspace{0.25cm}
\subsubsection{FLUDS $\alpha$-pass Local Portion} 

\textbf{Filling the lockbox phase}:\newline
In this phase we do a simplistic loop over faces. Note that during this phase we also determine the maximum degrees of freedom per face which will define our local storage requirement.
\begin{itemize}
\item Initially the lockbox (vector of pairs) is empty, the first cell is encountered. We also instantiate an outbound face counter (\xmltag{outgoing\_face\_count}) and a vector of integers denoting slot indices.
 (\xmltag{vector<int> outb\_face\_slot\_indices})
\item For each face index of the cell (which will never be a large number hence the use of \xmltag{short}) it is determined if the face is outgoing (dot product with the sweep ordering's reference direction vector). If it is, we push the cell's global index and the associated face index \textbf{as a pair} to the lockbox. We also increment the outbound face counter and assign a new slot index.
\item For each subsequent cell in the topological sorting we do the following:
\begin{itemize}
\item We first loop over all the incoming faces. From our meshing datastructure we will know the upstream neighbor of that face and because the neighbor is local we can determine (from the vertices) which face of the neighbor is associated with the current face. This provides us with a look up pair (global cell index and face index)
\item We then filter the lockbox with this pair as lookup and once we find the lockbox slot we empty it by assigning a negative number to the cell index.
\item We now loop over the outgoing faces but instead of simply pushing a new pair into the lockbox, we first filter through the lockbox looking for an empty slot. If we found an empty slot we place the current cell's global index and face index pair into this slot (as well as assign a new slot index). If we did not find an open slot we push into a new slot (and assign the slot index).
\end{itemize}

\end{itemize}

\textbf{Mapping local incident faces phase}:\newline
With the lockbox slotting strategy developed we now start a another phase. During the first phase we determined where outgoing faces will store their information. The storage information for each of those faces is determined by the first DOF of every outgoing face since the data will be contigous for face DOF $i=0...N$. The contigous nature does not apply to the downstream cell's incoming face since the face vertices are in reverse order (right hand rule) and their respective DOF$_0$ is not necessarily matched. We therfore need to fill the following data structure for each cell:\newline

\xmltag{vector<pair<int,vector<int>>> inco\_face\_dof\_mapping}\newline

The first level of this structure is a vector incremented by a incident face counter and then holding a pair comprising a lockbox slot and a vector. It is trivial to envision that we can determine an incident face's lockbox slot from the previously developed information, i.e. we know the neighboring cell's index and we can determine the associated face, therefore we can determine where that face will start storing its data (which will be a small contiguous block). We then just have to map each of the current cell's face DOFs to a location in this small contiguous block.

\begin{itemize}
\item For each cell in the topological sort we loop over only the incident faces. \xmltag{incoming\_face\_count} is incremented. We determine the face's upstream cell as well as its associated face. This determines the slot index.
\item We then do a mapping of the face DOFs. A fairly trivial operation by which we loop over the current cell face DOFs and internally loop over the upstream cell's face DOFs. If a match is found the mapping is pushed to the vector portion of the mapping pair.
\end{itemize}

At the end of this pass we will have two primary FLUDS data structures:
\newline

\xmltag{vector<vector<int>> so\_cell\_outb\_face\_slot\_indices} \newline 
Given a cell's sweep ordering index, then outbound face counter, this structure holds the location index of the start of the face's outbound storage information.\newline

\xmltag{vector<vector<pair>> so\_cell\_inco\_face\_dof\_indices} \newline 
Given a cell's sweep ordering index, then inbound face counter, this structure holds a pair. The first item of the pair is where the upstream information block starts and the second is a mapping of the face's DOFs inside this block.
\newline

These two data structures completely define where a cell can store its outgoing angular flux (per face) and also where it can retrieve its incoming angular flux (per face). Accessing these location within a sweep chunk is achieved with the two functions calls:


\vspace{2cm}
See next page

\newpage
\begin{lstlisting}[language=c++]
double*  chi_mesh::sweep_management::FLUDS::
 OutgoingPsi(int cell_so_index, int outb_face_counter,
              int face_dof, int n)
{
  int index =
    local_psi_Gn_block_stride*G*n +
    so_cell_outb_face_slot_indices[cell_so_index][outb_face_counter]*
    local_psi_stride*G +
    face_dof*G;

  return &ref_local_psi->operator[](index);
}
\end{lstlisting}

\begin{lstlisting}[language=c++]
double*  chi_mesh::sweep_management::FLUDS::
UpwindPsi(int cell_so_index, int inc_face_counter,
             int face_dof,int g, int n)
{
  int index =
    local_psi_Gn_block_stride*G*n +
    so_cell_inco_face_dof_indices[cell_so_index][inc_face_counter].first*
    local_psi_stride*G +
    so_cell_inco_face_dof_indices[cell_so_index][inc_face_counter].
      second[face_dof]*G + g;

  return &ref_local_psi->operator[](index);
}
\end{lstlisting}




\newpage
\renewcommand{\thefigure}{A\arabic{section}.\arabic{figure}}
\chead{Quadrature rules}
\begin{appendices}
\vspace{1cm}
\section{Quadrature rules for integration over angle-space}

Suppose we have a function of azimuthal angle $\varphi$ and polar angle $\theta$ namely, $f(\varphi,\theta)$, and we integrate this function over the entire angular domain. We seek a quadrature rule (or a combination of rules) that will allow the simplified integration in the form
\beq
\int_0^{2\pi}\int_0^\pi f(\varphi,\theta).sin\theta.d\theta.d\varphi\ 
\approx \  \sum_{n}^N w_n f((\varphi,\theta)_n).
\eeq 

Here the values $w_n$ are weights and $(\varphi,\theta)_n$ are the abscissae of the t.b.d. quadrature rule. However, the form of this integral is not yet in a form conducive to the application of quadrature rules. To this end we observe that 
$$
\mu = cos \theta
$$

and 
$$
\frac{d\mu}{d\theta} = -sin\theta
$$

We can now express $f$ as a function of $\mu$ instead of $\theta$ for which we have
\beq
&\int_0^{2\pi} \int_{1}^{-1}  -f(\varphi,\mu).d\mu.d\varphi \\
=&\int_0^{2\pi} \int_{-1}^{1}  f(\varphi,\mu).d\mu.d\varphi 
\eeq 

This equation is convenient since we can apply quadrature rules to each angular case. To see this let us assign 
\beqn
g(\varphi) = \int_{-1}^1  f(\varphi,\mu).d\mu
\eeqn

for which we now have 
\beqn\label{eq:integratevarphi}
\int_0^{2\pi} g(\varphi).d\varphi.
\eeqn

\vspace{1cm}
\subsection{Gauss-Legendre quadrature rule}
The \textit{Gauss-Legendre} quadrature rule is used when the weighting function is unity, i.e. $w(x){=}1$, and therefore we can integrate the function in the form
$$
\int_{-1}^1 f(x).dx \ \approx \sum_{n=1}^N w_n f(x_n).
$$

Here the abscissae $x_n$ are the roots of the Legendre polynomial $P_n(x)$, and the weights are

\beqn\label{eq:legendreweights}
w_n = \frac{2(1-x_n^2)}{(n+1)^2  (P_{n+1}(x_n))^2}
\eeqn
\newline
From this equation we can see we need to evaluate the values of the Legendre polynomials. For this purpose we use the recursion of Legendre polynomials stating that

\beqn
P_0(x)&=1 \\
P_1(x)&=x\\
P_{n+1}(x) &= \biggr(\frac{2n+1}{n+1}\biggr)x. P_n(x) -\biggr(\frac{n}{n+1}\biggr) P_{n-1}(x)
\eeqn

Using the code below these functions can be evaluated.

\begin{lstlisting}[language=python]
def Legendre(N,x):
    Pnm1 = 1;
    Pn   = x;
    Pnp1 = 0;
    
    if (N==0):
        return 1;
    
    if (N==1):
        return x;
    
    for n in range(2,N+1):
        ns=n-1
        Pnp1 = ((2*ns+1)/(ns+1))*x*Pn -(ns/(ns+1))*Pnm1;
        Pnm1 = Pn;
        Pn = Pnp1;
        
    return Pnp1 
\end{lstlisting}

The other unknown in equation \ref{eq:legendreweights} is the abscissae $x_n$ which are the roots of the Legendre polynomial equations. An algorithm for finding these roots is given by Barrerra-Figueroa \cite{roots}. This algorithm utilizes Newton's method for finding a root and therefore we need to also have a function for finding the derivate of the Legendre polynomials

\beqn
P_{n}'(x) &= \frac{nx}{x^2-1}. P_n(x) - \frac{n}{x^2-1}P_{n-1}(x)
\eeqn
\newline
The code below obtains the derivative $P_n'(x)$:
\begin{lstlisting}[language=python]
def dLegendredx(N,x):
    if (N==0):
        return 0;
    
    if (N==1):
        return 1;
    
    return (N*x/(math.pow(x,2)-1))*Legendre(N,x)- \
           (N/(math.pow(x,2)-1))*Legendre(N-1,x);
\end{lstlisting}

Finally applying the root finding equation in \cite{roots}
\beqn
x_k^{(\ell+1)} &=x_k^{(\ell)} - \frac{f(x_k^{(\ell)})}{f'(x_k^{(\ell)}) - f(x_k^{(\ell)}) \sum_{j=1}^{k-1} \frac{1}{x_k^{(\ell)} - x_j  }   }
\eeqn

We get the code

\begin{lstlisting}[language=python]
def LegendreRoots(N,maxiters=1000,tol=1.0e-10):
    xn = np.linspace(-0.999,0.999,N);  #Initial guessed values

    print("Initial guess:")
    print(xn)

    wn = np.zeros((N));
    
    
    for k in range(0,N):
        print("Finding root %d of %d" % (k+1,N), end='')
        i=0;
        while (i<maxiters):
            xold = xn[k]
            a = Legendre(N,xold)
            b = dLegendredx(N,xold)
            c = 0;
            for j in range(0,k):
                c=c+(1/(xold-xn[j]))
            
            xnew = xold - (a/(b-a*c))
            
            res=abs(xnew-xold)
            xn[k]=xnew

            if (res<tol):
                print('tr',end='')
                break
            i=i+1
        
        wn[k] = 2*(1-xn[k]*xn[k])/(N+1)/(N+1)/ \
                Legendre(N+1,xn[k])/Legendre(N+1,xn[k])
        print(" root %f, weight=%f, test=%f" %(xn[k],wn[k],Legendre(N,xn[k])))
        
    return xn,wn;
\end{lstlisting}

With the result for $P_5(x)$:
\begin{verbatim}
Finding root 1 of 5 root1 -0.906180, weight=0.236927
Finding root 2 of 5 root1 -0.538469, weight=0.478629
Finding root 3 of 5 root1 0.000000, weight=0.568889
Finding root 4 of 5 root1 0.538469, weight=0.478629
Finding root 5 of 5 root1 0.906180, weight=0.236927
\end{verbatim}

\vspace{1cm}
\subsection{Gauss-Chebyshev quadrature rule}
The \textit{Gauss-Chebyshev} quadrature rule is used when the weighting is $w(x)=\frac{1}{\sqrt{1-x^2}}$ and therefore the integral is of the form

$$
\int_{-1}^1 \frac{f(x)}{\sqrt{1-x^2}}.dx \approx \sum_{n=1}^N w_n f(x_n).
$$

Here the abscissae $x_n$ are the roots of the Chebyshev polynomials of the second kind which have an explicit solution:
\beqn
x_n = cos\biggr(\frac{(2n-1)\pi}{2N} \biggr).
\eeqn

The quadrature weights are given by the simple relation
\beqn
w_n = \frac{\pi}{N}.
\eeqn

The code below is a simple implementation of these formulas

\begin{lstlisting}[language=python]
def ChebyshevRoots(N):
    xn = np.linspace(-1,1,N)
    wn = np.linspace(-1,1,N)
    
    for n in range(0,N):
        ns=n+1
        xn[n]=math.cos((2*ns-1)*math.pi/2/N)
        wn[n]=math.pi/N
        
        print("Finding root %d of %d, root=%f, weight=%f" %(n+1,N,xn[n],wn[n]))
        
    return xn,wn
\end{lstlisting}
With the result for $U_5(x)$:
\begin{verbatim}
Finding root 1 of 5, root=0.951057, weight=0.628319
Finding root 2 of 5, root=0.587785, weight=0.628319
Finding root 3 of 5, root=0.000000, weight=0.628319
Finding root 4 of 5, root=-0.587785, weight=0.628319
Finding root 5 of 5, root=-0.951057, weight=0.628319
\end{verbatim}

\newpage
\subsection{Application to Discrete Ordinates}
We can find the polar angles associated with the polar quadrature directly from our earlier definition

\beq
g(\varphi) = \int_{-1}^1  f(\varphi,\mu).d\mu \approx \sum_{i=0}^{2N_p-1} w_i f(\varphi,\mu_i)
\eeq

where $N_p$ is the number of polar quadrature points/angles in the first octant. The abscissae $\mu_i$ are obtained from the roots of the Legendre polynomial $P_{2N_p}$ and the weights are 

\beqn
w_i = \frac{2(1-\mu_i^2)}{(i+2)^2  (P_{i+2}(\mu_i))^2}.
\eeqn

The angles associated with the abscissae are then 

\beqn
\theta_i = cos^{-1}\mu_i
\eeqn

\subsection{Gauss-Legendre-Legendre product quadrature}

The integration over all azimuthal angles requires some thought. In its defined form

\beq
\int_0^{2\pi} g(\varphi).d\varphi.
\eeq

it can utilize the Gauss-Legendre quadrature after a change of intervals from $[0,2\pi]$ to $[-1,1]$ as

\beq
\int_0^{2\pi} g(\varphi).d\varphi 
&= \frac{2\pi-0}{2}\int_{-1}^1 g(\frac{2\pi-0}{2} y + \frac{2\pi+0}{2}).dy \\
&= \pi \int_{-1}^1 g(\pi y + \pi).dy \\
&\approx  \sum_{j=0}^{4N_a-1} w_j g(\pi y_j+\pi)
\eeq

where again $N_a$ is the amount of azimuthal angles per octant and the quadrature points are the abscissae of the Legendre polynomial $P_{4N_a}$ and the weights are 

\beqn
w_j = \frac{2\pi(1-y_j^2)}{(j+2)^2  (P_{j+2}(y_j))^2}.
\eeqn

The angles associated with the abscissae are then 

\beqn
\varphi_j =\pi y_j + \pi
\eeqn

and we now have a product quadrature of the form

\beqn
\int_0^{2\pi}\int_0^\pi f(\varphi,\theta).sin\theta.d\theta.d\varphi\ 
\approx \  \sum_{j=0}^{4N_a-1} w_j \biggr[
\sum_{i=0} ^{2N_p-1}   w_i f(\varphi_j,\theta_i)
\biggr].
\eeqn
\newline
An example of this quadrature is shown for the first octant in Figure \ref{fig:s8gausslegendrelegendre} where the colors and dot size indicate the quadrature weight and the dot's position indicates the quadrature point on the unit sphere.
\begin{figure}[h]
\centering
\includegraphics[width=0.7\linewidth]{Gauss-Legendre-Legendre.png}
\caption{Quadrature points and weights (colors) for the Gauss-Legendre quadrature set for both the polar and azimuthal angles with $N_a=8$ and $N_p=8$.}
\label{fig:s8gausslegendrelegendre}
\end{figure}

\newpage
\subsection{Gauss-Legendre-Chebyshev product quadrature}
Instead of changing the intervals of the integration over the azimuthal angles $\varphi$ we can instead look towards utilizing these intervals by defining
$$
y = cos(\frac{\varphi}{2})
$$
and
$$
\frac{dy}{d\varphi} = -\frac{1}{2}sin(\frac{\varphi}{2}).
$$
Therefore
\beq
d\varphi &= - \frac{2}{sin(\frac{\varphi}{2})}.dy \\
&= - \frac{2}{\sqrt{  1 - cos^2 (\frac{\varphi}{2})    }}.dy\\
\therefore d\varphi &= - \frac{2}{\sqrt{  1 - y^2     }}.dy
\eeq

which can be used in equation \ref{eq:integratevarphi} as

\beq
\int_0^{2\pi} g(\varphi).d\varphi 
&= \int_1^{-1} -2 \frac{g(2cos^{-1}y)}{\sqrt{  1 - y^2     }}.dy \\
&=2 \int_{-1}^{1}  \frac{g(2cos^{-1}y)}{\sqrt{  1 - y^2     }}.dy \\
\eeq

and when we define, for simplicity, $h(y)=g(2cos^{-1}y)$ we get a familiar form

\beqn
\int_0^{2\pi} g(\varphi).d\varphi = 2\int_{-1}^1 \frac{h(y)}{ \sqrt{1-y^2}  }.dy.
\eeqn
\newline
This equation can be approximated with a Gauss-Chebyshev quadrature with abscissae
\beqn
y_j = cos\biggr(  \frac{(2j+1)\pi}{8N_a}  \biggr)
\eeqn

and equal weights
\beqn
w_j = \frac{2\pi}{4N_a}
\eeqn

where $N_a$ is the number of quadrature points per octant, to get the quadrature formula

\beqn
\int_{-1}^1 \frac{h(y)}{ \sqrt{1-y^2}  }.dy \approx \  \sum_{j=0}^{4N_a-1} w_j h(y_j).
\eeqn
\newline
The angles associated with the abscissae are then 
\beqn
\varphi_j = \frac{(2j+1)\pi}{8N_a} 
\eeqn

and we now have a product quadrature of the form

\beqn
\int_0^{2\pi}\int_0^\pi f(\varphi,\theta).sin\theta.d\theta.d\varphi\ 
\approx \  \sum_{j=0}^{4N_a-1} w_j \biggr[
\sum_{i=0} ^{2N_p-1}   w_i f(\varphi_j,\theta_i)
\biggr].
\eeqn
\newline
An example of this quadrature is shown for the first octant in Figure \ref{fig:s8gausslegendrechebyshev} where the colors and dot size indicate the quadrature weight and the dot's position indicates the quadrature point on the unit sphere.
\begin{figure}[h]
\centering
\includegraphics[width=0.7\linewidth]{Gauss-Legendre-Chebyshev.png}
\caption{Quadrature points and weights (colors) for the Gauss-Legendre quadrature for the polar integration and the Gauss-Chebyshev quadrature for the azimuthal angles with $N_a=8$ and $N_p=8$.}
\label{fig:s8gausslegendrechebyshev}
\end{figure}


\newpage
\subsection{Evaluation of product quadratures}
With the quadrature candidates established we can now look at their performance. The application of the product quadratures is done to calculate the integral
\begin{equation*}
\int_{4\pi} \Psi_g(r,\hat{\Omega}) Y_{\ell m^*}(\hat{\Omega}).d\hat{\Omega} 
\end{equation*}

where the angular flux $\Psi(r,\hat{\Omega})$ is known at the point where the integration is to be performed. Hence we are essentially looking for a function to integrate the spherical harmonic $Y_{lm^*}(\hat{\Omega})$ where, repeated here

$$Y_{\ell m*}(\theta ,\varphi ) = (-1)^mY_{\ell (-m)}(\theta ,\varphi )$$

\begin{equation}
Y_{\ell m} (\theta, \varphi )=
\begin{cases}
(-1)^m \sqrt(2)\sqrt{ \frac{(2\ell + 1)}{4\pi}   \frac{(\ell-|m|)!}{(\ell+|m|)!}}P_{\ell}^{|m|}(cos\theta)sin\ {|m|\varphi}
& \text{if } m < 0 \\
(-1)^m \sqrt(2)\sqrt{ \frac{(2\ell + 1)}{4\pi}   \frac{(\ell-m)!}{(\ell+m)!}}P_{\ell}^{m}(cos\theta)cos\ {m\varphi}
& \text{if } m \ge 0 \\
\end{cases}
\end{equation}
\newline
where the associated Legendre polynomials, $P_\ell^m$ can be determined from

\beqn
P_0^0 &= 1, \quad \quad \quad
P_1^{0} = x, \\
P_\ell^\ell &= - (2\ell-1) \sqrt{1-x^2} \ P_{\ell-1}^{\ell-1}(x) \quad \text{ and}\\
(\ell - m)
P_\ell^m &= (2\ell-1)x \ P_{\ell-1}^m(x) - (\ell+m -1)P_{\ell-2}^m (x).
\eeqn
\newline
An algorithm implementation to evaluate these associated Legendre polynomials is shown below.

\begin{lstlisting}[language=python]
def AssociatedLegendre(ell,m,x):
    if (abs(m)>ell):
        return 0;
    
    #====m=0,l=1
    Pn   = x;
    
    #====m=1,l=1
    Pnpos= -math.sqrt(1-math.pow(x,2));
    #====m=-1,l=1
    Pnneg= -0.5*Pnpos;
    
    #====m=0,l=0
    if (ell==0):
        return 1;
    
    if (ell==1):
        if (m==-1):
            return Pnneg;
        if (m==0):
            return Pn;
        if (m==1):
            return Pnpos;

    Pmlp1 = 0
    if (ell==m):
        Pmlp1 = -(2*ell-1)*math.sqrt(1-math.pow(x,2.0))* \
                AssociatedLegendre(ell-1,ell-1,x)
    else:
        Pmlp1 = (2*ell-1)*x*AssociatedLegendre(ell-1,m,x);
        Pmlp1 = Pmlp1 - (ell+m-1)*AssociatedLegendre(ell-2,m,x)
        Pmlp1 = Pmlp1 / (ell-m)
    
    return Pmlp1 
\end{lstlisting} 

An algorithm implementation of the tesseral spherical harmonics is shown below.

\begin{lstlisting}[language=python]
#============================
def fac(x):
    if (x==0):
        return 1
    if (x==1):
        return 1
    
    return fac(x-1)*x;

#============================
def Min1powerm(m):
    if (m==0):
        return 1;
    if ((m%2)==0):
        return 1
    else:
        return -1

#============================    
def Ylm(ell,m,theta,varphi):
    if (m<0):
        return Min1powerm(m)*math.sqrt( \
               ( (2*ell+1)/(2*math.pi) )* \
               fac(ell-abs(m))/fac(ell+abs(m)) )* \
               AssociatedLegendre(ell,abs(m),math.cos(theta))* \
               math.sin(abs(m)*varphi)
    else:
        return Min1powerm(m)*math.sqrt( \
               ( (2*ell+1)/(2*math.pi) )* \
               fac(ell-m)/fac(ell+m) )* \
               AssociatedLegendre(ell,m,math.cos(theta))* \
               math.cos(m*varphi)
\end{lstlisting} 




\newpage 
\renewcommand{\thefigure}{B\arabic{section}.\arabic{figure}}
\section{More on Spherical Harmonics}
\subsection{Expansion of a function of two angles $f(\varphi,\theta)$}
We seek to expand an angularly dependent function $f(\varphi,\theta)$. Why? Don't know yet, but one such expansion from fundamental theory is spherical harmonics:
\begin{equation} \label{eq:harmonicExpansion}
f(\varphi,\theta) = \sum_{\ell=0}^{\infty}\sum_{m=-\ell}^{\ell}   f_{\ell }^m Y_{\ell }^m(\varphi,\theta  ).
\end{equation}
\newline
This expansion will be very unfamiliar for engineers since most of the scientific computing in common disciplines deal with cartesian coordinates. Unfortunately it will be nearly impossible to explain the development of an expansion of an angular function in spherical harmonics without first exploring the means to calculate its unknowns. To this end let us begin with stating that \textbf{there are two flavors of spherical harmonics}. The regular form $Y_{\ell}^m$, which contains complex numbers (challenging to handle), and a real form $Y_{\ell m}$.
The unknowns in equation \ref{eq:harmonicExpansion} has a trail of components the first of which is
\begin{equation*}
\begin{aligned}
f_{\ell }^m
&=\int _{\hat{\Omega} }f(\varphi,\theta )\,Y_{\ell }^{m^*}(\varphi,\theta)\,d\hat{\Omega} \\
&=\int _{0}^{2\pi }\int _{0}^{\pi } f(\varphi,\theta )Y_{\ell }^{m^*}(\varphi,\theta ) .\sin \theta .d\theta .d\varphi .
\end{aligned}
\end{equation*}

The reader should try to comprehend that the $f_{\ell}^m$ components are almost never computed in this form since doing so means one already has a analytical representation of $f(\varphi,\theta)$. Additionally we have

\begin{equation}
Y_{\ell}^m (\varphi,\theta)= 
(-1)^m \sqrt{
\frac{(2\ell+1)}{4\pi}
\frac{(\ell-m)!}{(\ell+m)!}
}
.P_{\ell}^m (\cos \theta)e^{ (m \varphi)i}
\end{equation}
and its complex conjugate
$$Y_{\ell }^{m^*}(\varphi,\theta) = (-1)^mY_{\ell }^{(-m)}(\varphi,\theta).$$
This form of the spherical harmonics functions can be very unruly and therefore its more common place to calculate them from the real forms as

\begin{equation}
Y_{\ell}^m (\varphi,\theta)=
\begin{cases}
\frac{1}{\sqrt{2}} (Y_{\ell |m|}- i Y_{\ell,-|m|}) & \text{if }m<0 \\
Y_{\ell0} & \text{if } m=0 \\
\frac{(-1)^m}{\sqrt{2}} (Y_{\ell |m|}+ i Y_{\ell,-|m|}) & \text{if }m>0 \\
\end{cases}
\end{equation}
Here the real forms are represented by:
\begin{equation}\label{eq:Ylm}
Y_{\ell m} (\theta, \varphi )=
\begin{cases}
(-1)^m \sqrt(2)\sqrt{ \frac{(2\ell + 1)}{4\pi}   \frac{(\ell-|m|)!}{(\ell+|m|)!}}P_{\ell}^{|m|}(\cos\theta)sin\ {|m|\varphi}
& \text{if } m < 0 \\
\sqrt{ \frac{(2\ell + 1)}{4\pi}} P_{\ell}^{m}(cos\theta) & \text{if } m = 0 \\
(-1)^m \sqrt(2)\sqrt{ \frac{(2\ell + 1)}{4\pi}   \frac{(\ell-m)!}{(\ell+m)!}}P_{\ell}^{m}(\cos\theta)cos\ {m\varphi}
& \text{if } m \ge 0 \\
\end{cases}
\end{equation}
\newline
Finally the associated Legendre polynomials, $P_\ell^m$ can be determined from

\beqn 
P_0^0 &= 1, \quad \quad \quad
P_1^{0} = x, \\
P_\ell^\ell &= - (2\ell-1) \sqrt{1-x^2} \ P_{\ell-1}^{\ell-1}(x) \quad \text{ and}\\
(\ell - m)
P_\ell^m &= (2\ell-1)x \ P_{\ell-1}^m(x) - (\ell+m -1)P_{\ell-2}^m (x).
\eeqn
\newline
With all these unknows we see that before we can calculate the expansion we need to choose the maximum order $L=\ell_{max}$ after which we need to compute each $P_\ell^m$, each $Y_{\ell m}$, each $Y_{\ell}^m$ and each $Y_\ell^{m^*}$. Only then can we compute $f_{\ell}^m$. If we do all the work this way we can approximate some functions.

\subsection{Prototype code for spherical harmonics}
We begin with code to compute the associated Legendre polynomials

\begin{lstlisting}[language=python]
def AssociatedLegendre(ell,m,x):
    if (abs(m)>ell):
        return 0;
    
    #====m=0,l=1
    Pn   = x;
    
    #====m=1,l=1
    Pnpos= -math.sqrt(1-math.pow(x,2));
    #====m=-1,l=1
    Pnneg= -0.5*Pnpos;
    
    #====m=0,l=0
    if (ell==0):
        return 1;
    
    if (ell==1):
        if (m==-1):
            return Pnneg;
        if (m==0):
            return Pn;
        if (m==1):
            return Pnpos;

    Pmlp1 = 0
    if (ell==m):
        Pmlp1 = -(2*ell-1)*math.sqrt(1-math.pow(x,2.0))* \
                AssociatedLegendre(ell-1,ell-1,x)
    else:
        Pmlp1 = (2*ell-1)*x*AssociatedLegendre(ell-1,m,x);
        Pmlp1 = Pmlp1 - (ell+m-1)*AssociatedLegendre(ell-2,m,x)
        Pmlp1 = Pmlp1 / (ell-m)
    
    return Pmlp1 
\end{lstlisting}

\newpage
We then depict code to calculate the real forms of the spherical harmonics

\begin{lstlisting}[language=python]
#============================
def fac(x):
    if (x==0):
        return 1
    if (x==1):
        return 1
    
    return fac(x-1)*x;

#============================
def Min1powerm(m):
    if (m==0):
        return 1;
    if ((m%2)==0):
        return 1
    else:
        return -1

#============================    
def Ylm(ell,m,varphi,theta):
    if (m<0):
        return Min1powerm(m)*math.sqrt( \
               ( (2*ell+1)/(2*math.pi) )* \
               fac(ell-abs(m))/fac(ell+abs(m)) )* \
               AssociatedLegendre(ell,abs(m),math.cos(theta))* \
               math.sin(abs(m)*varphi)
    elif (m==0):
        return math.sqrt( \
               ( (2*ell+1)/(4*math.pi) )* \
               fac(ell-m)/fac(ell+m) )* \
               AssociatedLegendre(ell,m,math.cos(theta))* \
               math.cos(m*varphi)
    else:
        return Min1powerm(m)*math.sqrt( \
               ( (2*ell+1)/(2*math.pi) )* \
               fac(ell-m)/fac(ell+m) )* \
               AssociatedLegendre(ell,m,math.cos(theta))* \
               math.cos(m*varphi)
\end{lstlisting}

And then the complex form of the spherical harmonics
\begin{lstlisting}[language=python]
#============================
def Yl_m(ell,m,varphi,theta):
    v = 0+0j;
    if (m<0):
        v = (1/math.sqrt(2))*complex(Ylm(ell,abs(m),varphi,theta), - \
                                     Ylm(ell,-abs(m),varphi,theta))
    elif (m==0):
        v = complex(Ylm(ell,0,varphi,theta),0)
    else:
        v = Min1powerm(m)* \
        (1/math.sqrt(2))*complex(Ylm(ell,abs(m),varphi,theta), \
                                 Ylm(ell,-abs(m),varphi,theta))
        
    return v
\end{lstlisting}

\newpage
From here we need to have a means to compute the integral

\begin{equation*}
\begin{aligned}
f_{\ell }^m
&=\int _{\hat{\Omega} }f(\varphi,\theta )\,Y_{\ell }^{m^*}(\varphi,\theta)\,d\hat{\Omega} \\
&=\int _{0}^{2\pi }\int _{0}^{\pi } f(\varphi,\theta )Y_{\ell }^{m^*}(\varphi,\theta ) .\sin \theta .d\theta .d\varphi .
\end{aligned}
\end{equation*}

Suppose we have any function of angle \texttt{F0}, \texttt{F1}, \texttt{F2} and so forth we can define a function that will add the complex conjugate of the spherical harmonics with the simple code

\begin{lstlisting}[language=python]
#============================
def Flm(ell,m,varphi,theta):
    return F2(varphi,theta)*Min1powerm(m)* \
           (Legendre.Yl_m(ell,-m,varphi,theta))
\end{lstlisting}

We can then precompute a vector containing all the $f_{\ell m}$ coefficients using either a Riemann integral or a quadrature rule

\begin{lstlisting}[language=python]
GLC = Legendre.Quadrature()
GLC.InitializeWithGLC(8,8)


#=========================== Build flm
L =7
k=-1
flm = np.zeros((L*(L+2)+1),dtype=np.complex_)
for ell in range(0,L+1):
    for m in range(-ell,ell+1):
        k=k+1
        #flm[k] = Legendre.RiemannAngLM(F1lm,ell,m)
        
        #print("l=%f, m=%f, flm=" %(ell,m), end='')
        #print(flm[k])
        flm[k] = Legendre.QuadratureIntegrateLM(Flm,GLC,ell,m)
        print("l=%f, m=%f, flm=" %(ell,m), end='')
        print(flm[k])
\end{lstlisting}

Finally we can compute a representation of the expansion over the span of $\varphi$ using the code

\begin{lstlisting}[language=python]
#============================ Build yi1
Ni1=200
varphi1=np.linspace(0,2*math.pi,Ni1)
yi1=np.zeros((Ni1))

for i in range(0,Ni1):
    yi1[i]= 0;
    v = 0+0j
    k=-1;
    for ell in range(0,L+1):
        for m in range(-ell,ell+1):
            k=k+1
            v=v+(flm[k])* \
            (Legendre.Yl_m(ell,m,varphi1[i],math.pi/2))
            yi1[i] = v.real
\end{lstlisting}

\subsubsection{Approximating an isotropic flux}
The simplest function to represent is an isotropic flux (i.e. $f(\varphi,\theta)=1$). Such a function is perfectly capture with $L=0$, i.e. a single expansion, as shown in Figure \ref{fig:ylmf0}. This is not surprising since the combination of spherical harmonics with order and moment zero results in $\sqrt{\frac{1}{4\pi}} \stimes \sqrt{\frac{1}{4\pi}}$ which integrates to unity and hence the original function is recovered.
\begin{lstlisting}[language=python]
def F0(varphi,theta):
    return 1
\end{lstlisting}
\begin{figure}[H]
\centering
\includegraphics[width=0.8\linewidth]{Ylm_f0}
\caption{Approximation of a pure isotropic function with spherical harmonics. The plot is shown for the azimuthal angle $\varphi$ only.}
\label{fig:ylmf0}
\end{figure}

\subsubsection{Approximating an anisotropic but smooth flux}
We can construct an anisotropic function of azimuthal angle as
$$
f(\varphi,\theta) = 1+\cos (4\varphi)
$$

\begin{lstlisting}[language=python]
def F1(varphi,theta):
    return 1.0+0.1*math.cos(varphi*4)
\end{lstlisting}

Such a function requires a few more orders of spherical harmonics in order to capture the shape and, as shown in Figure \ref{fig:ylmf7}, $L=7$ closely resembles the shape. A function of this shape could appear in a fuel assembly lattice where the effective scattering and absorbtion is a strong function of azimuthal angle.

\begin{figure}[H]
\centering
\includegraphics[width=0.8\linewidth]{Ylm_f7}
\caption{Approximation of an anisotropic smooth function with spherical harmonics. The plot is shown for the azimuthal angle $\varphi$ only. The radial dimension represents the flux magnitude.}
\label{fig:ylmf7}
\end{figure}

\newpage
\subsubsection{Approximating a directional flux (i.e. anisotropic + not-smooth)}
As a final consideration we try to construct a function that is very angular, like a beam. Such a function of angle could be
\begin{equation*}
f(\varphi,\theta) = 
\begin{cases}
\frac{2}{10} & \text{if } \quad \varphi<\frac{7}{8}\pi \\
\frac{2}{10} + \frac{6}{5}\cos (4\varphi) & \text{if }\quad \frac{7}{8}\pi \le \varphi \le \frac{9}{8}\pi\\
\frac{2}{10} & \text{if }\quad \ \varphi>\frac{9}{8}\pi \\
\end{cases}
\end{equation*}

\begin{lstlisting}[language=python]
def F2(varphi,theta):
    if (varphi<(7*math.pi/8)):
        return 0.2;
    if (varphi>(9*math.pi/8)):
        return 0.2;
    
    return 1.2*math.cos(varphi*4)+0.2
\end{lstlisting}

As expected a total number of 12 spherical harmonic orders are required to accurately represent such a directional flux (see Figure \ref{fig:ylmf12}). An additional 2D plot is shown in Figure \ref{fig:ylmf12b} which more clearly shows the oscillations of the expansion at the directions not aligned with the directional nature of the function.

\begin{figure}[H]
\centering
\includegraphics[width=0.8\linewidth]{Ylm_f12}
\caption{Approximation of an anisotropic smooth function with spherical harmonics. The plot is shown for the azimuthal angle $\varphi$ only. The radial dimension represents the flux magnitude.}
\label{fig:ylmf12}
\end{figure}

\begin{figure}[H]
\centering
\includegraphics[width=0.8\linewidth]{Ylm_f12b}
\caption{Approximation of an anisotropic smooth function with spherical harmonics. The plot is shown for the azimuthal angle $\varphi$ only. }
\label{fig:ylmf12b}
\end{figure}

\subsection{Prototype code for real form of the spherical harmonics}
For the real form of the spherical harmonics we have a slightly modified real form 

\beqn
Y_{\ell m} (\theta, \varphi )=
\begin{cases}
 \sqrt(2)\sqrt{ \frac{(2\ell + 1)}{4\pi}   \frac{(\ell-|m|)!}{(\ell+|m|)!}}P_{\ell}^{|m|}(\cos\theta)sin\ {|m|\varphi}
& \text{if } m < 0 \\
\sqrt{ \frac{(2\ell + 1)}{4\pi}} P_{\ell}^{m}(cos\theta) & \text{if } m = 0 \\
 \sqrt(2)\sqrt{ \frac{(2\ell + 1)}{4\pi}   \frac{(\ell-m)!}{(\ell+m)!}}P_{\ell}^{m}(\cos\theta)cos\ {m\varphi}
& \text{if } m \ge 0 \\
\end{cases}
\eeqn

and the expansion coefficients are also different

\beqn 
f(\varphi,\theta) = \sum_{\ell = 0}^\infty \sum_{m=-\ell}^{\ell} f_{\ell m} Y_{\ell m}(\varphi,\theta)
\eeqn 

where
$$
f_{\ell m} = \int_{0}^{2\pi} \int_0^\pi f(\varphi,\theta)Y_{\ell m}(\varphi,\theta).sin\theta.d\theta.d\varphi
$$

for which the code is

\begin{lstlisting}[language=python]
#============================    
def Ylmcoeff(ell,m,varphi,theta):
    if (m<0):
        return math.sqrt( \
               ( (2*ell+1)/(2*math.pi) )* \
               fac(ell-abs(m))/fac(ell+abs(m)) )* \
               AssociatedLegendre(ell,abs(m),math.cos(theta))* \
               math.sin(abs(m)*varphi)
    elif (m==0):
        return math.sqrt( \
               ( (2*ell+1)/(4*math.pi) )* \
               fac(ell-m)/fac(ell+m) )* \
               AssociatedLegendre(ell,m,math.cos(theta))* \
               math.cos(m*varphi)
    else:
        return math.sqrt( \
               ( (2*ell+1)/(2*math.pi) )* \
               fac(ell-m)/fac(ell+m) )* \
               AssociatedLegendre(ell,m,math.cos(theta))* \
               math.cos(m*varphi)
\end{lstlisting}

And

\begin{lstlisting}[language=python]
def Flm(ell,m,varphi,theta):
    return F3(varphi,theta)* \
           (Legendre.Ylmcoeff(ell,m,varphi,theta))
           
#=========================== Build flm
L =12
k=-1
flm = np.zeros((L*(L+2)+1),dtype=np.complex_)
for ell in range(0,L+1):
    for m in range(-ell,ell+1):
        k=k+1
        #flm[k] = Legendre.RiemannAngLM(F1lm,ell,m)
        
        #print("l=%f, m=%f, flm=" %(ell,m), end='')
        #print(flm[k])
        flm[k] = Legendre.QuadratureIntegrateLM(Flm,GLC,ell,m)
        print("l=%f, m=%f, flm=" %(ell,m), end='')
        print(flm[k])

#============================ Build yi1
Ni1=400
varphi1=np.linspace(0,2*math.pi,Ni1)
yi1=np.zeros((Ni1))

for i in range(0,Ni1):
    yi1[i]= 0;
    v = 0
    k=-1;
    for ell in range(0,L+1):
        for m in range(-ell,ell+1):
            k=k+1
            v=v+(flm[k])* \
            (Legendre.Ylmcoeff(ell,m,varphi1[i],math.pi/2))
            yi1[i] = v
\end{lstlisting}



\newpage 
\renewcommand{\thefigure}{C.\arabic{figure}}
\section{Creating simple materials for testing}
Neutral particle transport involves three basic processes:
\begin{itemize}
\item Absorption. The elimination of the particle from a current group
\item Scattering. Change in angle and group essentially removing the particle from the group-angle pair.
\item Source. Both in the form of a fixed source and as reactions to absorption processes (i.e. (n,n), (n,2n), (n,fission), etc.)
\end{itemize}

We also have a fundamental definition that the total removal process is the sum of the absorption process and the scattering process. In terms of nuclide cross-sections we can write this as

$$
\sigma_t = \sigma_a + \sigma_s 
$$
Where $\sigma_t$, $\sigma_a$ and $\sigma_s$ represent the total-, absorption- and scattering cross-sections respectively.

\chead{Scattering kinematics}
\subsection{Simple neutron-nuclide scattering processes}

\begin{figure}[H]
\centering
\includegraphics[width=0.85\linewidth]{LatexDraw/CollisionKinematics1}
\caption{Collision kinematics of a stationary nuclide in both the laboratory reference frame and the center-of-mass reference frame.}
\label{fig:collisionkinematics1}
\end{figure}


Let us consider a statyionary target nucleus $X_Z^A$ suspended in space and a neutron moving towards this nucleus (from left to right) at velocity $\text{v}_{L}$, where $L$ denotes the laboratory reference frame (i.e. the one we are living in) as denoted in Figure \ref{fig:collisionkinematics1}. Assuming the target nucleas is at rest (an invalid assumption that will be treated later) with velocity $V_L$ we know the energies associated with these particles to be 
\beq
E_L &= \frac{1}{2} m \text{v}_L^2 \\
E_{L_A} &= \frac{1}{2} M \cancelto{0}{V_L^2}
\eeq
where $E_L$ is the energy of the neutron and $E_{L_A}$ is the energy of the target nucleus, both in the laboratory reference frame, and $m=1$ is the mass of the neutron and $M=A$ is the mass of the target nucleus. Fortunately the derivation of the mass-momentum equations relating the center-of-mass energies and angles to the laboratory reference frame quantities are comprehensively depicted in the textbook by Duderstadt and Hamilton \cite{Duderstadt}. In this book the scattering angle in the laboratory reference frame, $\theta_L$, is related to the scattering angle in the center-of-mass reference frame, $\theta_c$, as

\beqn \label{eq:thetaLvsthetac}
\tan \theta_L = \frac{  \sin \theta_c    } {  \frac{1}{A} +\cos \theta_c     }.
\eeqn 
Associated with this, \cite{Duderstadt} also derives the neutron energy change $E_L \to E_L'$ as 

\beqn 
E_L' = \biggr[
\frac{  (1+\alpha) + (1-\alpha)\cos \theta_c   }{2}
\biggr] E_L
\eeqn 

where $\alpha = ( \frac{A-1}{A+1})^2$. For light nuclei, where one can assume the scattering angle in the center-of-mass frame is isotropic \cite{Duderstadt} we can determine the probability distribution function for scattering through an angle $\theta_L$. From equation \ref{eq:thetaLvsthetac} we can get $\theta_L$ as 

\beqn \label{eq:thetaLvsthetac2} 
\theta_L = 
\begin{cases}
\pi + 
\cos^{-1} \biggr(
\frac{\sin \theta_c}{ \frac{1}{A} + \cos \theta_c      }
\biggr)
& \text{if } (\frac{1}{A} + \cos \theta_c) < 0 \\
\frac{\pi}{2} & \text{if } (\frac{1}{A} + \cos \theta_c) = 0 \\
\cos^{-1} \biggr(
\frac{\sin \theta_c}{ \frac{1}{A} + \cos \theta_c      }
\biggr)
& \text{if } (\frac{1}{A} + \cos \theta_c) > 0 \\
\end{cases}
\eeqn 
\newline
Since $\cos \theta_c$, $\theta_c \in [0,\pi]$, is essentially our cumulative probability when linearly mapped to $[0,1]$ we require the inverse of equation \ref{eq:thetaLvsthetac}. Therefore we begin by setting $x=\cos \theta_c$ and inserting it into equation \ref{eq:thetaLvsthetac2}

\beq 
\tan \theta_L = \frac{ \sqrt{1-x^2}      }{ \frac{1}{A} + x}
\eeq 

 and for simplicity we set the unknowns to constants
 \beq 
C &= \frac{ \sqrt{1-x^2}      }{ B + x} \\
C(B+x) &= \sqrt{1-x^2}  \\
C^2(x^2+2Bx + B^2) &= 1-x^2 \\
C^2x^2+2BC^2x + B^2C^2 &= 1 -x^2 \\
(C^2+1) x^2 +2BC^2 x + B^2C^2 -1 &= 0
 \eeq 
  which is now in the familiar form $ax^2 + bx +c =0$ for which we complete the square to find
  
 \beq
 x &= \frac{  -2BC^2   \pm \sqrt{  4B^2C^4 - 4(C^2+1)(B^2C^2-1)    }     }{2(C^2+1)}\\
 \therefore x&= \frac{  -2BC^2   \pm 2C\sqrt{  1-B^2+\frac{1}{C^2}   }     }{2(C^2+1)}.
 \eeq
The two possible $x$ values obtained this way was incurred because we applied a square to remove the square-root term and therefore will give us the angle corresponding to $\tan \theta_L$ as well as $-\tan \theta_L$. Since we know that we started with a positive $\tan \theta_L$ we are only interested in the solution

 \beqn
 \cos \theta_c =  x = f(\theta_L)=g(\mu) &= \frac{  -2BC^2  + 2C\sqrt{  1-B^2+\frac{1}{C^2}   }     }{2(C^2+1)} .
 \eeqn
 
 And therefore our mapping to a cumulative probability distribution becomes
 
 \beqn 
\int_{-1}^1 P(\mu).d\mu = \frac{g(\mu)-1}{2} 
 \eeqn 
 where $B{=}1/A$ and $C{=} \tan ( \cos^{-1} \mu)$. The cumulative probability function for different masses of the target nucleus is shown in Figure \ref{fig:scatanglevsa}. Obtaining the probability density function, $P(\mu)$ is then a simple differentiation that can be done numerically 
 
\beqn
P(\mu)=\frac{1}{2} \frac{dg}{d\mu}
\eeqn  
for which the results are shown in Figure \ref{fig:scatproblevsa}. The algorithm applied to find $P(\mu)$ is shown below

\begin{lstlisting}[language=python]
#============================
def ThetaC(thetaL,A):
    B = 1/A
    if (thetaL == (math.pi/2)):
        C=1
    else:
        C = min(math.tan(thetaL),1.0e6)
    
    root = (1-B**2+1/(C**2))
    
    x1 = (-2*B*C**2 + 2*C*math.sqrt(root))/2/(C**2+1)
    #x2 = (-2*B*C**2 - 2*C*math.sqrt(root))/2/(C**2+1)
    
    #Safety catches
    if (x1>1.0):
        return math.acos(1)
    if (x1<-1.0):
        return math.acos(-1)
    
    return math.acos(x1)
\end{lstlisting}

\newpage
\begin{lstlisting}[language=python]
#============================ Probability scattering mu
def Pmu(mu,A):
    thetaLA = math.acos(mu-0.0000001)
    thetaLB = math.acos(mu+0.0000001)
    thetacA = ThetaC(thetaLA,A)
    thetacB = ThetaC(thetaLB,A)
    CPLA    = 0.5-0.5*math.cos(thetacA)
    CPLB    = 0.5-0.5*math.cos(thetacB)
    return -(CPLB - CPLA)/0.0000002
\end{lstlisting}



\begin{figure}[H]
\centering
\includegraphics[width=0.7\linewidth]{ScatAnglevsA}
\caption{Cumulative probability distribution for a neutron scattering off a stationary nucleus of mass A.}
\label{fig:scatanglevsa}
\end{figure}

\begin{figure}[H]
\centering
\includegraphics[width=0.7\linewidth]{ScatProbvsA}
\caption{Probability distribution for a neutron scattering off a stationary nucleus of mass A.}
\label{fig:scatproblevsa}
\end{figure}



\subsection{Combining probabilities}
Now, since $\mu$ corresponds to a discrete $\theta_c$ which also corresponds to a discrete $\frac{E_L'}{E_L}$, the kernel value $K(\mu,E'\to E)$ also will have only discrete points where it is non-zero, i.e.

\beq
K(\mu,E' \to E)=
\begin{cases}
P(\mu)P(E'\to E)    &, \text{if } E' - \biggr[
\frac{  (1+\alpha) + (1-\alpha) g(\mu)   }{2}
\biggr] E = 0 \\
0 &, otherwise
\end{cases}
\eeq 

This discrete behavior requires significant numerical effort to resolve, however, multi-group integrations of the source- and destination energy groups alleviates this somewhat since

\beq 
K(\mu,E_{g'} \to E_g) &=
\int_{E_{g+1}'}^{E_g'} \int_{E_{g+1}}^{E_g} K(\mu,E' \to E).dE.dE' \\
&=P(\mu)P(E_{g'}\to E_g) 
\eeq 
\newline
An algorithm to implement this kernel is shown below
\newpage
\begin{lstlisting}[language=python]
def Kernel(mu,gprime,g,Eg,A,Ng=1000):
    #=== Bin boundaries
    Eiupp = Eg[gprime]
    Eilow = Eg[gprime+1]
    Efupp = Eg[g]
    Eflow = Eg[g+1]
    
    dEi = (Eiupp - Eilow)/Ng
    binWidth = (Efupp-Eflow)
    
    sumprobs=0
    thetaL = math.acos(mu)
    thetac = ThetaC(thetaL,A)
    muc = math.cos(thetac)
    for iE in range(0,Ng):
        Ein = Eilow + dEi/2 + dEi*iE        
        Eout=Ef(muc,A,Ein)
        
        if ((Eout<=Efupp) and (Eout>=Eflow)):
            sumprobs = sumprobs + 1/Ng
    
    return sumprobs*Pmu(mu,A)
\end{lstlisting}

In order to test the multi-group implementation of this kernel we can build a simple 10 group energy discretization $E\in [0,1]$ MeV with linearly spaced bins

\begin{lstlisting}[language=python]
G = 10
Eg = np.linspace(1,0,G+1)
\end{lstlisting}

The requirement here is that the continuous form obeys

\beq 
\int_{-1}^1 \int_0^\infty K(\mu,E' \to E).dE.d\mu = 1
\eeq 

and therefore the multi-group form must obey
\beq 
\int_{-1}^1 \biggr( \sum_{g=0}^G  K(\mu,E_{g'} \to E_g) \biggr).d\mu = 1.
\eeq 
 The code to implement this integration is
 
\begin{lstlisting}[language=python]
Np=100
mu=np.linspace(-0.9999,0.9999,Np)
ydis=np.zeros((Np))
sumofdis=0
sumovergroupsdis=0
for i in range(0,Np):
    for gdes in range(0,G):
        sumovergroupsdis=sumovergroupsdis+ \
                         Kernel(mu[i],gprime,gdes,Eg,A)*(2/Np)
\end{lstlisting}

and proves that the integral is unity as intended. Another test is to integrate over all angle with

\beq 
&\int_{4\pi} \biggr( \sum_{g=0}^G  K(\Omega' \bigcdot \Omega,E_{g'} \to E_g) \biggr).d\Omega' \\
=&\int_{0}^{2\pi} \int_0^\pi  
\biggr( \sum_{g=0}^G  K(\Omega' \bigcdot \Omega,E_{g'} \to E_g) \biggr)
.\sin \theta' .d\theta'.d\varphi' \\
=&2\pi
\eeq 

where $\Omega' = [\sin\theta' . \cos \varphi', \sin\theta' . \sin \varphi', \cos \theta']$ and $\Omega$ is chosen arbitrarily (i.e. $\Omega=[1,0,0]$). The code to compute this integral, using the previous denoted 10-group energy structure, as well as scattering from group 0, is shown below

\begin{lstlisting}[language=python]
Np = 100
Na = 200
polar = np.linspace(0.0001,math.pi*0.9999,Np)
azimu = np.linspace(0.0001,2*math.pi*0.99999,Na)
dtheta = (math.pi)/Np
dvarphi= (math.pi*2)/Na

nref = np.array([1.0,0.0,0.0])
ndir = np.array([0.0,0.0,0.0])

sumprob=0.0
gprime=0
for i in range(0,Na):
    print(i)
    for j in range(0,Np):
        varphi = azimu[i]
        theta  = polar[j]
        
        ndir[0] = math.sin(theta)*math.cos(varphi)
        ndir[1] = math.sin(theta)*math.sin(varphi)
        ndir[2] = math.cos(theta)
        
        mu = np.dot(ndir,nref)
        
        for gdes in range(0,G):
            sumprob=sumprob+ \
              Kernel(mu,gprime,gdes,Eg,A)*math.sin(theta)*dtheta*dvarphi
\end{lstlisting}

Indeed this does then integrate to $2\pi$.

\newpage
\subsection{Legendre expansion of the scattering term}
The discrete ordinates method involves the expansion of the scattering term using Legendre polynomials as basis functions. This expansion is of the form

\begin{align*}
K(\mu,E_{g'} {\to} E_g) &= \sum_{\ell=0}^\infty \frac{2\ell+1}{2} P_\ell (\mu) K_\ell (E_{g'} {\to }E_g)
\end{align*}
where the expansion coefficients are given by
\begin{align*}
K_\ell(E_{g'} {\to} E_g) &= \int_{-1}^1 K(\mu,E_{g'} {\to} E_g).P_\ell (\mu).d\mu
\end{align*}

The code to compute the expansion coefficients requires just a small modification of the multi-group kernel in the sense that the Kernel is multiplied by the Legendre polynomial. The code is shown below

\begin{lstlisting}[language=python]
def KernelL(ell,gprime,g,Eg,A):
    groupprob=0
    Np=800
    mu=np.linspace(-0.9999,0.9999,Np)
    dmu = (0.9999*2)/Np
    print("Integrating group %d to %d moment %d" %(gprime,g,ell))
    for i in range(0,Np):
        groupprob = groupprob + Kernel(mu[i],gprime,g,Eg,A)* \
             Legendre.Legendre(ell,mu[i])*dmu
    
    return groupprob
\end{lstlisting}

We can now decide to truncate our expansion at the $L$-th moment and precompute the expansion coefficient $K_\ell(E_{g'} {\to} E_g)$ with an example scattering from group $g'=0$ to group $g=1$ the code is

\begin{lstlisting}[language=python]
L=10
KL=np.zeros((L+1))
gprime = 0
g=1
sumgroups=0
for ell in range(0,L+1):
    KL[ell] = KernelL(ell,gprime,g,Eg,A)
\end{lstlisting}

We can now compute the discrete and expanded form as a function of $\mu$ for which the code is shown below. A graphical plot of the expanded and discrete form is shown, for different orders of expansion, in Figure  \ref{fig:kernelg0tog1}.
\newpage
\begin{lstlisting}[language=python]
#============================ Discrete form vs Expansion for Group 0 to 1        
Np=100
mu=np.linspace(-0.9999,0.9999,Np)
yexp=np.zeros((Np))
ydis=np.zeros((Np))
sumofdis=0
sumofexp=0
sumovergroupsdis=0
for i in range(0,Np):
    yexp[i] = expansion(L,mu[i],KL)
    ydis[i] = Kernel(mu[i],gprime,g,Eg,A)
    sumofexp=sumofexp+yexp[i]*(2/Np)
    sumofdis=sumofdis+ydis[i]*(2/Np)
    
    for gdes in range(0,G):
        sumovergroupsdis=sumovergroupsdis+ \
                         Kernel(mu[i],gprime,gdes,Eg,A)*(2/Np)
\end{lstlisting}

\begin{figure}[H]
\centering
\includegraphics[width=0.7\linewidth]{KernelG0toG1}
\caption{Kernel function for a neutron scattering off of a stationary carbon nuclear ($A=12$) and scattering from group 0 to 1.}
\label{fig:kernelg0tog1}
\end{figure}




\newpage
\renewcommand{\thefigure}{D\arabic{section}.\arabic{figure}}
\chead{Quadrature rules for integration of triangle space}
\section{Quadrature rule for integration of triangle space} \label{appendix:trianglequadrature}
We seek an integral of a function in triangle space $T_{sp}$ in the form

\beq 
\int \int_{T_{sp}} f(x,y).dx.dy = \sum_{i=0}^{N-1} w_i f(x_i,y_i).
\eeq 

Furthermore we know that in the finite element method with only linear shape functions we will at most have polynomials of degree 2 therefore we can devise a set of test functions

\beq 
&f(x,y) = 1    &&\int_{0}^1 \int_0^{1-y} 1.dx.dy = \frac{1}{2} = \sum_{i=0}^{N-1} w_i\\
&f(x,y) = x    &&\int_{0}^1 \int_0^{1-y} x.dx.dy = \frac{1}{6} = \sum_{i=0}^{N-1} w_i x_i\\
&f(x,y) = y    &&\int_{0}^1 \int_0^{1-y} y.dx.dy = \frac{1}{6} = \sum_{i=0}^{N-1} w_i y_i\\
&f(x,y) = xy  & &\int_{0}^1 \int_0^{1-y} xy.dx.dy = \frac{1}{24} = \sum_{i=0}^{N-1} w_i x_i y_i\\
&f(x,y) = x^2    &&\int_{0}^1 \int_0^{1-y} x^2.dx.dy = \frac{1}{12} = \sum_{i=0}^{N-1} w_i x_i^2\\
&f(x,y) = y^2    &&\int_{0}^1 \int_0^{1-y} y^2.dx.dy = \frac{1}{12} = \sum_{i=0}^{N-1} w_i y_i^2\\
\eeq 

With $N=3$ a symmetric solution is obtained with

\beq
w_i &= \frac{1}{6} \\
x_0,y_0 &= ( \frac{1}{6}, \frac{1}{6}) \\
x_1,y_1 &= ( \frac{4}{6}, \frac{1}{6}) \\
x_2,y_2 &= ( \frac{1}{6}, \frac{4}{6}) \\
\eeq 
which is not a unique solution.


\newpage
\renewcommand{\thefigure}{D\arabic{section}.\arabic{figure}}
\chead{Quadrature rules for integration of tetrahedron space}
\section{Quadrature rule for integration of tetrahedron space} \label{appendix:tetrahedronquadrature}
The study of quadratures for tetrahedrons is a deeply mathematical topic one that is outside the scope of this study. As with the two dimensional case, and since we will limit our study to piece-wise linear shape functions we will limit our quadrature set to a minimum degree of precision of 2. Meaning we only need to exactly integrate polynomials of to the second degree. For tetrahedons, in natural coordinates, quadrature sets are available in \cite{quadraturerulestet}. For this study the weights and quadrature points as shown in Table below will be used.

\begin{table}[H] \label{tbl:qpointstet}
\centering
\begin{tabular}{|l|l|l|l|l|}
\hline
\textbf{Point} & \textbf{weights} & \textbf{X}  & \textbf{Y}  & \textbf{Z}  \\ \hline
0              & 0.25             & 0.585410197 & 0.138196601 & 0.138196601 \\ \hline
1              & 0.25             & 0.138196601 & 0.138196601 & 0.138196601 \\ \hline
2              & 0.25             & 0.138196601 & 0.138196601 & 0.585410197 \\ \hline
3              & 0.25             & 0.138196601 & 0.585410197 & 0.138196601 \\ \hline
\end{tabular}
\caption{Quadrature points and weights used for tetrahedron elements.}
\end{table}

\end{appendices}

\newpage
\chead{References}
\begin{thebibliography}{1}
    
    \bibitem{roots} Barrera-Figueroa V., et al. {\em Multiple root finder algorithm for Legendre and Chebyshev polynomials via Newton’s method}, Annales Mathematicae et Informaticae, volume 33, pages 3-13, 2006
    
    \bibitem{Lewis} Lewis E.E, Miller W.F. {\em Computational Methods of Neutron Transport}, John Wiley \& Sons, 1984, ISBN 0-471-09245-2
    
    \bibitem{Duderstadt} Duderstadt J.J., Hamilton L.J., {\em Nuclear Reactor Analysis}, John Wiley \& Sons, 1976.
    
     \bibitem{BaileyAdamsPWLPolygons}  Bailey T.S., Chang J.H., Adams M.L., {\em A Piecewise Linear Discontinous Finite Element spatial discretization of the transport equation in 2D Cylindrical Geometry}, 2009 International Conference on Advances in Mathematics, Computational Methods, and Reactor Physics, 2008.
    
    \bibitem{BaileyAdamsPWBLPolygons}  Bailey T.S., Warsa J.S., Chang J.H., Adams M.L., {\em A Piecewise Bi-linear Discontinous Finite Element spatial discretization of the S$_n$ transport equation}, International Conference on Mathematics and Computational Methods Applied to Nuclear Science and Engineering, 2011.
    
    \bibitem{MathisFunMatrixInverse} Pierce, Rod. {\em Inverse of a Matrix using Minors, Cofactors and Adjugate} Math Is Fun. Ed. Rod Pierce. 22 Nov 2018. 1 Jan 2019 http://www.mathsisfun.com/algebra/matrix-inverse-minors-cofactors-adjugate.html
    
    \bibitem{quadraturerulestet} Engels H., Zienkiewicz O., {\em Quadrature Rules for Tetrahedrons}, http://people.sc.fsu.edu/~jburkardt/datasets/quadrature\_rules\_tet/quadrature\_rules\_tet.html, accessed January 1, 2019.
    
    \bibitem{BaileyAdamsPolyhedral} Bailey T.S., Adams M.L., Yang B., Zika M.R., {\em A piecewise linear finite element discretization of the diffusion equation for arbitrary polyhedral grids}, Journal of Computational Physics 227 (2008) 3738–3757, 2007
    
    \bibitem{delaunay} Cheng et al, {\em Delaunay Mesh Generation}, Chapman \& Hall/CRC Computer \& Information Science Series, 2013
    
    \bibitem{AdamsSPDS} Adams M.P., Hawkins W.D., Adams M.L., {\em Managing Information Flow in Graph Traversals}, Texas A\&M University, November 2018
    
    
\end{thebibliography}





\end{document}